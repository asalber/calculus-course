% !TEX root = ../calculus_manual.tex
\section{Differential calculus with one variable}
%---------------------------------------------------------------------slide----
\begin{frame}
\frametitle{Differential calculus with one variable}
\tableofcontents[sectionstyle=show/hide,hideothersubsections]
\end{frame}


\subsection{Concept of derivative}
%---------------------------------------------------------------------slide----
\begin{frame}
\frametitle{Increment}
\begin{definition}[Increment of a variable]
An \emph{increment} of a variable $x$ is a change in the value of the variable and is denoted $\Delta x$.
The increment of a variable $x$ along an interval $[a,b]$ is
\[
\Delta x = b-a.
\]
\end{definition}

\begin{definition}[Increment of a function]
The \emph{increment} of a function $y=f(x)$ along an interval $[a,b]\subseteq Dom(f)$ is
\[
\Delta y = f(b)-f(a).
\]
\end{definition}

\textbf{Example} The increment of $x$ along the interval $[2,5]$ is $\Delta x=5-2=3$ and the increment of the function $y=x^2$ along the same interval is $\Delta y=5^2-2^2=21$.
\end{frame}

\begin{frame}
\frametitle{Average rate of change}
The study of a function $y=f(x)$ requires to anderstand how the function changes, that is, how changes the dependent variable $y$ when we change the independent variable $x$.

\begin{definition}[Average rate of change]
The  \emph average rate of change of a function $f$ in an interval $[a,a+\Delta x]\subseteq Dom(f)$, is the quotient between the increment of $y=f(x)$ and the increment of $x$ in that interval, and is denoted
\[
\mbox{ARC}\;f[a,a+\Delta x]=\frac{\Delta y}{\Delta x}=\frac{f(a+\Delta x)-f(a)}{\Delta x}.
\]
\end{definition}
\end{frame}

% %---------------------------------------------------------------------slide----
% \begin{frame}
% \frametitle{Tasa de variación media: Ejemplo}
% Consideremos la función $y=x^2$ que mide el área de un cuadrado de chapa metálica de lado $x$.
%
% Si en un determinado instante el lado del cuadrado es $a$, y sometemos la chapa a un proceso de calentamiento que aumenta el lado del cuadrado una cantidad $\Delta x$, ¿en cuánto se incrementará el área del cuadrado?
% \begin{columns}
% \begin{column}{0.3\textwidth}
% \begin{align*}
% \Delta y &= f(a+\Delta x)-f(a)=(a+\Delta x)^2-a^2=\\
% &= a^2+2a\Delta x+\Delta x^2-a^2=2a\Delta x+\Delta x^2.
% \end{align*}
% \end{column}
% \begin{column}{0.3\textwidth}
% \begin{center}
% \scalebox{1}{\input{img/calculo_diferencial_1_variable/variacion_area_cuadrado}}
% \end{center}
% \end{column}
% \end{columns}
% ¿Cuál será la tasa de variación media del área en el intervalo $[a,a+\Delta x]$?
% \[
% \textrm{TVM}\;f[a,a+\Delta x]=\frac{\Delta y}{\Delta x}=\frac{2a\Delta x+\Delta x^2}{\Delta x}=2a+\Delta x.
% \]
% \end{frame}

%
% %---------------------------------------------------------------------slide----
% \begin{frame}
% \frametitle{Interpretación geométrica de la tasa de variación media}
% La tasa de variación media de $f$ en el intervalo $[a,a+\Delta x]$ es la pendiente de la recta \emph{secante} a $f$ en los puntos $(a,f(a))$ y $(a+\Delta x,f(a+\Delta x))$.
% \begin{center}
% \scalebox{1}{\input{img/calculo_diferencial_1_variable/secante}}
% \end{center}
% \end{frame}
%
%
% %---------------------------------------------------------------------slide----
% \begin{frame}
% \frametitle{Tasa de variación instantánea}
% En muchas ocasiones, es interesante estudiar la tasa de variación que experimenta una función, no en intervalo, sino en
% un punto.
%
% Conocer la tendencia de variación de una función en un instante puede ayudarnos a predecir valores en instantes
% próximos.
%
% \begin{definicion}[Tasa de variación instantánea y derivada]
% Dada una función $y=f(x)$, se llama \emph{tasa de variación instantánea} de $f$ en un punto $a$, al límite de la tasa de
% variación media de $f$ en el intervalo $[a,a+\Delta x]$, cuando $\Delta x$ tiende a 0, y lo notaremos
% \[
% \textrm{TVI}\;f (a)=\lim_{\Delta x\rightarrow 0} \textrm{TVM}\; f[a,a+\Delta x]=\lim_{\Delta x\rightarrow 0}\frac{\Delta y}{\Delta x}=\lim_{\Delta x\rightarrow 0}\frac{f(a+\Delta x)-f(a)}{\Delta x}
% \]
% Cuando este límite existe, se dice que la función $f$ es derivable en el punto $a$, y al valor del mismo se le llama
% derivada de $f$ en $a$, y se nota como
% \[
% f'(a) \mbox{ o bien } \frac{df}{dx}(a)
% \]
% \end{definicion}
% \end{frame}
%
%
% %---------------------------------------------------------------------slide----
% \begin{frame}
% \frametitle{Tasa de variación instantánea: Ejemplo}
% Consideremos de nuevo la función $y=x^2$ que mide el área de un cuadrado de chapa metálica de lado $x$.
%
% Si en un determinado instante el lado del cuadrado es $a$, y sometemos la chapa a un proceso de calentamiento que
% aumenta el lado del cuadrado, ¿cuál es la tasa de variación instantánea del área del cuadrado en dicho instante?
% \begin{align*}
% \textrm{TVI}\;f(a)]&=\lim_{\Delta x\rightarrow 0}\frac{\Delta y}{\Delta x}=\lim_{\Delta x\rightarrow 0}\frac{f(a+\Delta x)-f(a)}{\Delta x} =\\
% &=\lim_{\Delta x\rightarrow 0}\frac{2a\Delta x+\Delta x^2}{\Delta x}=\lim_{\Delta x\rightarrow 0} 2a+\Delta x= 2a.
% \end{align*}
% Así pues,
% \[
% f'(a)=2a,
% \]
% lo que indica que la tendencia de crecimiento el área es del doble del valor del lado.
%
% El signo de $f'(a)$ indica la tendencia de crecimiento de $f$ en el punto $a$:
% \begin{itemize}
% \item[--]  $f'(a)>0$ indica que la tendencia es creciente.
% \item[--]  $f'(a)<0$ indica que la tendencia es decreciente.
% \end{itemize}
% \end{frame}
%
%
% %---------------------------------------------------------------------slide----
% \begin{frame}
% \frametitle{Interpretación geométrica de la tasa de variación instantánea}
% La tasa de variación instantánea de $f$ en el punto $a$ es la pendiente de la recta \emph{tangente} a $f$ en el punto $(a,f(a))$.
% \begin{center}
% \scalebox{1}{\input{img/calculo_diferencial_1_variable/tangente}}
% \end{center}
% \end{frame}
%
%
%
% \subsection{Álgebra de derivadas}
% %---------------------------------------------------------------------slide----
% \begin{frame}
% \frametitle{Propiedades de la derivada}
% Si $y=c$, es una función constante, entonces $y'=0$.
%
% Si $y=x$, es la función identidad, entonces  $y'=1$.
%
% Si $u=f(x)$ y $v=g(x)$ son dos funciones diferenciables, entonces
% \begin{itemize}
% \item $(u+v)'=u'+v'$
% \item $(u-v)'=u'-v'$
% \item $(u\cdot v)'=u'\cdot v+ u\cdot v'$
% \item $\left(\dfrac{u}{v}\right)'=\dfrac{u'\cdot v-u\cdot v'}{v^2}$
% \end{itemize}
% \end{frame}
%
%
% \subsection{Derivada de una función compuesta: La regla de la cadena}
%
% %---------------------------------------------------------------------slide----
% \begin{frame}
% \frametitle{Diferencial de una función compuesta}
% \framesubtitle{La regla de la cadena}
% \begin{teorema}[Regla de la cadena] Si
% $y=f\circ g$ es la composición de dos funciones $y=f(z)$ y $z=g(x)$, entonces
% \[
% (f\circ g)'(x)=f'(g(x))g'(x),
% \]
% \end{teorema}
%
% Resulta sencillo demostrarlo con la notación diferencial
% \[
% \frac{dy}{dx}=\frac{dy}{dz}\frac{dz}{dx}=f'(z)g'(x)=f'(g(x))g'(x).
% \]
%
% \structure{\textbf{Ejemplo}} Si $f(z)=\sen z$ y $g(x)=x^2$, entonces $f\circ g(x)=\sen(x^2)$ y, aplicando la regla de la cadena, su derivada
% vale
% \[
% (f\circ g)'(x)=f'(g(x))g'(x) = \cos g(x) 2x = \cos(x^2)2x.
% \]
% Por otro lado, $g\circ f(z)= (\sin z)^2$ y, de nuevo aplicando la regla de la cadena, su derivada vale
%
% \[
% (g\circ f)'(z)=g'(f(z))f'(z) = 2f(z)\cos z = 2\sen z\cos z.
% \]
% \end{frame}
%
%
%
% \subsection{Derivada de la inversa de una función}
% %---------------------------------------------------------------------slide----
% \begin{frame}
% \frametitle{Derivada de la función inversa}
% \begin{teorema}[Derivada de la inversa]
% Si $y=f(x)$ es una función y $x=f^{-1}(y)$ es su inversa, entonces
% \[
% \left(f^{-1}\right)'(y)=\frac{1}{f'(x)}=\frac{1}{f'(f^{-1}(y))}
% \]
% \end{teorema}
%
% También resulta sencillo de demostrar con la notación diferencial
% \[
% \frac{dx}{dy}=\frac{1}{dy/dx}=\frac{1}{f'(x)}=\frac{1}{f'(f^{-1}(y))}
% \]
%
% \structure{\textbf{Ejemplo}} La inversa de la función exponencial $y=f(x)=e^x$ es el logaritmo neperiano $x=f^{-1}(y)=\ln y$, de modo que
% para calcular la derivada del logaritmo podemos utilizar el teorema de la derivada de la inversa y se tiene
% \[
% \left(f^{-1}\right)'(y)=\frac{1}{f'(x)}=\frac{1}{e^x}=\frac{1}{e^{\ln y}}=\frac{1}{y}.
% \]
% \end{frame}
%
%
% \subsection{Interpretación cinemática de la derivada}
% % ---------------------------------------------------------------------slide----
% \begin{frame}
% \frametitle{Interpretación cinemática de la tasa de variación}
% \framesubtitle{Movimiento rectilineo}
% Supongase que la función $f(t)$ describe la posición de un objeto móvil sobre la recta real en el instante $t$.
% Tomando como referencia el origen de coordenadas $O$ y el vector unitario $\mathbf{i}=(1)$, se puede representar la
% posición $P$ del móvil en cada instante $t$ mediante un vector $\vec{OP}=x\mathbf{i}$ donde $x=f(t)$.
% \begin{center}
% \scalebox{1}{\input{img/calculo_diferencial_1_variable/movimiento_rectilineo}}
% \end{center}
%
% \structure{\textbf{Observación}}
% También tiene sentido pensar en $f$ como una función que mide otras magnitudes como por ejemplo la temperatura de un
% cuerpo, la concentración de un gas o la cantidad de un compuesto en una reacción química en un instante $t$.
% \end{frame}
%
%
% % ---------------------------------------------------------------------slide----
% \begin{frame}
% \frametitle{Interpretación cinemática de la tasa de variación media}
% En este contexto, si se toman los instantes $t=t_0$ y $t=t_0+\Delta t$, ambos del dominio $I$ de $f$, el vector
% \[
% \mathbf{v}_m=\frac{f(t_0+\Delta t)-f(t_0)}{\Delta t}
% \]
% que se conoce como \emph{velocidad media} de la trayectoria $f$ entre los instantes $t_0$ y $t_0+\Delta t$.
%
% \structure{\textbf{Ejemplo}}
% Un vehículo realiza un viaje de Madrid a Barcelona.
% Sea $f$ la función que da la posición el vehículo en cada instante.
% Si el vehículo parte de Madrid (km 0) a las 8 y llega a Barcelona (km 600) a las 14 horas, entonces la velocidad media
% del vehículo en el trayecto es
% \[
% \mathbf{v}_m=\frac{f(14)-f(8)}{14-8}=\frac{600-0}{6} = 100 km/h.
% \]
% \end{frame}
%
%
% % ---------------------------------------------------------------------slide----
% \begin{frame}
% \frametitle{Interpretación cinemática de la derivada}
% Siguiendo en este mismo contexto del movimiento rectilineo, la derivada de $f$ en el instante $t=t_0$ es el vector
% \[
% \mathbf{v}=f'(t_0)=\lim_{\Delta t\rightarrow 0}\frac{f(t_0+\Delta t)-f(t_0)}{\Delta t},
% \]
% que se conoce, siempre que exista el límite, como \emph{velocidad instantánea} o simplemente la \emph{velocidad} de la
% trayectoria $f$ en el instante $t_0$.
%
% Es decir, la derivada de la posición respecto del tiempo, es un campo de vectores que recibe el nombre de
% \emph{velocidad a lo largo de la trayectoria $f$}.
%
% \structure{\textbf{Ejemplo}}
% Siguiendo con el ejemplo anterior, lo que marca el velocímetro en un determinado instante sería el módulo del vector
% velocidad  en ese instante.
% \end{frame}
%
%
% % ---------------------------------------------------------------------slide----
% \begin{frame}
% \frametitle{Generalización al movimiento curvilineo}
% La derivada como velocidad a lo largo de una trayectoria en la recta real puede generalizarse a trayectorias en cualquier
% espacio euclídeo $\mathbb{R}^n$.
%
% Para el caso del plano real $\mathbb{R}^2$, si $f(t)$ describe la posición de un objeto móvil en el plano en el instante
% $t$, tomando como referencia el origen de coordenadas $O$ y los vectores coordenados
% $\{\mathbf{i}=(1,0),\mathbf{j}=(0,1)\}$, se puede representar la posición $P$ del móvil en cada instante $t$ mediante un
% vector $\vec{OP}=x(t)\mathbf{i}+y(t)\mathbf{j}$ cuyas coordenadas
% \[
% \begin{cases}
% x=x(t)\\
% y=y(t)
% \end{cases}
% \quad
% t\in I\subseteq \mathbb{R}
% \]
% se conocen como \emph{funciones coordenadas} de $f$ y se escribe $f(t)=(x(t),y(t))$.
%
% \begin{center}
% \scalebox{0.8}{\input{img/calculo_diferencial_1_variable/movimiento_curvilineo}}
% \end{center}
% \end{frame}
%
%
% % ---------------------------------------------------------------------slide----
% \begin{frame}
% \frametitle{Velocidad en una trayectoria curvilinea en el plano}
% En este contexto de una trayectoria $f(t)=(x(t),y(t))$ en el plano real $\mathbb{R}^2$, para un instante $t=t_0$, si existe el vector
% \[
% \mathbf{v} = \lim_{\Delta t\rightarrow 0} \frac{f(t_0+\Delta t)-f(t_0)}{\Delta t},
% \]
% entonces $f$ es derivable en el instante $t=t_0$ y el vector $\mathbf{v}=f'(t_0)$ se conoce como \emph{velocidad} de $f$ en ese instante.
%
% Como $f(t_0)=(x(t),y(t))$,
% \begin{align*}
% f'(t)&=\lim_{\Delta t\rightarrow 0} \frac{f(t_0+\Delta t)-f(t_0)}{\Delta t} = \lim_{\Delta t\rightarrow 0} \frac{(x(t_0+\Delta t),y(t_0+\Delta t))-(x(t_0),y(t_0))}{\Delta t} =\\
% &=  \lim_{\Delta t\rightarrow 0} \left(\frac{x(t_0+\Delta t)-x(t_0)}{\Delta t},\frac{y(t_0+\Delta t)-y(t_0)}{\Delta t}\right) =\\
% &= \left(\lim_{\Delta t\rightarrow 0}\frac{x(t_0+\Delta t)-x(t_0)}{\Delta t},\lim_{\Delta t\rightarrow 0}\frac{y(t_0+\Delta t)-y(t_0)}{\Delta t}\right) =
% (x'(t_0),y'(t_0)).
% \end{align*}
% luego
% \[
% \mathbf{v} = x'(t_0)\mathbf{i}+y'(t_0)\mathbf{j}.
% \]
% \end{frame}
%
%
% % ---------------------------------------------------------------------slide----
% \begin{frame}
% \frametitle{Velocidad en una trayectoria curvilinea en el plano}
% \framesubtitle{Ejemplo}
% Dada la trayectoria $f(t) = (\cos t,\sen t)$, $t\in \mathbb{R}$, cuya imagen es la circunferencia de centro el origen
% de coordenas y radio 1, sus funciones coordenadas son $x(t) = \cos t$, $y(t) = \sen t$, $t\in \mathbb{R}$, y su velocidad es
% \[
% \mathbf{v}=f'(t)=(x'(t),y'(t))=(-\sen t, \cos t).
% \]
% En el instante $t=\pi/4$, el móvil estará en la posición $f(\pi/4) = (\cos(\pi/4),\sen(\pi/4)) =(\sqrt{2}/2,\sqrt{2}/2)$
% y se moverá con una velocidad $\mathbf{v}=f'(\pi/4)=(-\sen(\pi/4),\cos(\pi/4))=(-\sqrt{2}/2,\sqrt{2}/2)$.
% \begin{center}
% \scalebox{0.8}{\input{img/calculo_diferencial_1_variable/circunferencia}}
% \end{center}
% Obsérvese que el módulo del vector velocidad siempre será 1 ya que
% $|\mathbf{v}|=\sqrt{(-\sen t)^2+(\cos t)^2}=1$.
% \end{frame}
%
%
%
% \subsection{Recta tangente a una trayectoria}
% % ---------------------------------------------------------------------slide----
% \begin{frame}
% \frametitle{Recta tangente a una trayectoria en el plano}
% Los vectores paralelos a la velocidad $\mathbf{v}$ se denominan \emph{vectores tangentes} a la trayectoria
% $f$ en el instante $t=t_0$, y la recta que pasa por $P=f(t_0)$ dirigida por $\mathbf{v}$ es la recta tangente a $f$ cuando
% $t=t_0$.
% \begin{definicion}[Recta tangente a una trayectoria]
% Dada una trayectoria $f$ sobre el plano real $\mathbb{R}^2$, se llama \emph{recta tangente} a $f$ para $t=t_0$ a la
% recta de ecuación
% \[
% l: (x,y)= f(t_0)+tf'(t_0) = (x(t_0),y(t_0))+t(x'(t_0),y'(t_0)) = (x(t_0)+tx'(t_0),y(t_0)+ty'(t_0)).
% \]
% \end{definicion}
% \structure{\texbf{Ejemplo}}
% Se ha visto que para la trayectoria $f(t) = (\cos t,\sen t)$, $t\in \mathbb{R}$, cuya imagen es la circunferencia de
% centro el origen de coordenas y radio 1, en el instante $t=\pi/4$ la posición del móvil era
% $f(\pi/4)=(\sqrt{2}/2,\sqrt{2}/2)$ y su velocidad $\mathbf{v}=(-\sqrt{2}/2,\sqrt{2}/2)$, de modo que la recta tangente a
% $f$ en ese instante es
% \[
% l: X=f(\pi/4)+t\mathbf{v} =
% \left(\frac{\sqrt{2}}{2},\frac{\sqrt{2}}{2}\right)+t\left(\frac{-\sqrt{2}}{2},\frac{\sqrt{2}}{2}\right) =
% \left(\frac{\sqrt{2}}{2}-t\frac{\sqrt{2}}{2},\frac{\sqrt{2}}{2}+t\frac{\sqrt{2}}{2}\right).
% \]
% \end{frame}
%
%
% % ---------------------------------------------------------------------slide----
% \begin{frame}
% \frametitle{Recta tangente a una trayectoria en el plano}
% De la ecuación vectorial de la recta tangente a $f$ para $t=t_0$, se obtiene que sus funciones cartesianas son
% \[
% \begin{cases}
% x=x(t_0)+tx'(t_0)\\
% y=y(t_0)+ty'(t_0)
% \end{cases}
% \quad t\in \mathbb{R},
% \]
% y despejando $t$ en ambas ecuaciones e igualando se llega a la ecuación cartesiana de la recta tangente
% \[
% \frac{x-x(t_0)}{x'(t_0)}=\frac{y-y(t_0)}{y'(t_0)},
% \]
% si $x'(t_0)\neq 0$ e $y'(t_0)\neq 0$, y de ahí a la ecuación en la forma punto-pendiente
% \[
% y-y(t_0)=\frac{y'(t_0)}{x'(t_0)}(x-x(t_0)).
% \]
% \structure{\texbf{Ejemplo}}
% Partiendo de la ecuación vectorial de la tangente del ejemplo anterior
% $l=\left(\frac{\sqrt{2}}{2}-t\frac{\sqrt{2}}{2},\frac{\sqrt{2}}{2}+t\frac{\sqrt{2}}{2}\right)$, su ecuación cartesiana
% es
% \[
% \frac{x-\sqrt{2}/2}{-\sqrt{2}/2} = \frac{y-\sqrt{2}/2}{\sqrt{2}/2}\Rightarrow y-\sqrt{2}/2 =
% \frac{-\sqrt{2}/2}{\sqrt{2}/2}(x-\sqrt{2}/2) \Rightarrow y=-x+\sqrt{2}.
% \]
% \end{frame}
%
%
% % ---------------------------------------------------------------------slide----
% \begin{frame}
% \frametitle{Recta normal a una trayectoria en el plano}
% Se ha visto que la recta tangente a una trayectoria $f$ cuando $t=t_0$ es la recta que pasa por el punto el punto
% $P=f(t_0)$ dirigida por el vector velocidad $\mathbf{v}=f'(t_0)=(x'(t_0),y'(t_0))$. Si en lugar de tomar ese vector se
% toma como vector director el vector $\mathbf{w}=(y'(t_0),-x'(t_0))$, que es ortogonal a $\mathbf{v}$, se obtiene otra
% recta que se conoce como \emph{recta normal} a la trayectoria $f$ cuanto $t=t_0$.
% \begin{definicion}[Recta normal a una trayectoria]
% Dada una trayectoria $f$ sobre el plano real $\mathbb{R}^2$, se llama \emph{recta normal} a $f$ para $t=t_0$ a la recta de ecuación
% \[
% l: (x,y)=(x(t_0),y(t_0))+t(y'(t_0),-x'(t_0)) = (x(t_0)+ty'(t_0),y(t_0)-tx'(t_0)).
% \]
% \end{definicion}
% Su ecuación cartesiana es
% \[
% \frac{x-x(t_0)}{y'(t_0)} = \frac{y-y(t_0)}{-x'(t_0)},
% \]
% y su ecuación en la forma punto pendiente
% \[
% y-y(t_0) = \frac{-x'(t_0)}{y'(t_0)}(x-x(t_0)).
% \]
% La recta normal es perpendicular a la recta tangente ya que sus vectores directores son ortogonales.
% \end{frame}
%
%
% % ---------------------------------------------------------------------slide----
% \begin{frame}
% \frametitle{Recta normal a una trayectoria en el plano}
% \framesubtitle{Ejemplo}
% Siguiendo con el ejemplo de la trayectoria $f(t) = (\cos t,\sen t)$, $t\in \mathbb{R}$, la recta normal en el instante $t=\pi/4$ es
% \begin{align*}
% l&: (x,y)=(\cos(\pi/2),\sen(\pi/2))+t(\cos(\pi/2),\sen(\pi/2)) =\\
% & \left(\frac{\sqrt{2}}{2},\frac{\sqrt{2}}{2}\right)+t\left(\frac{\sqrt{2}}{2},\frac{\sqrt{2}}{2}\right)
% =\left(\frac{\sqrt{2}}{2}+t\frac{\sqrt{2}}{2},\frac{\sqrt{2}}{2}+t\frac{\sqrt{2}}{2}\right),
% \end{align*}
% y su ecuación cartesiana es
% \[
% \frac{x-\sqrt{2}/2}{\sqrt{2}/2} = \frac{y-\sqrt{2}/2}{\sqrt{2}/2}\Rightarrow y-\sqrt{2}/2 = \frac{\sqrt{2}/2}{\sqrt{2}/2}(x-\sqrt{2}/2) \Rightarrow y=x.
% \]
% \begin{center}
% \scalebox{0.8}{\input{img/calculo_diferencial_1_variable/circunferencia_tangente_normal}}
% \end{center}
% \end{frame}
%
%
% % ---------------------------------------------------------------------slide----
% \begin{frame}
% \frametitle{Rectas tangente y normal a una función}
% Un caso particular de las recta tangente y normal a una trayectoria es son la recta tangente y normal a una función de una variable real.
% Si se tiene la función $y=f(x)$, $x\in I\subseteq \mathbb{R}$, una trayectoria que traza la gráfica de $f$ es
% \[
% g(t) = (t,f(t))  \quad t\in I,
% \]
% y su velocidad es
% \[
% g'(t) = (1,f'(t)),
% \]
% de modo que la recta tangente para $t=x_0$ es
% \[
% \frac{x-x_0}{1} = \frac{y-f(x_0)}{f'(x_0)} \Rightarrow y-f(x_0) = f'(x_0)(x-x_0),
% \]
% y la recta normal es
% \[
% \frac{x-x_0}{f'(x_0)} = \frac{y-f(x_0)}{-1} \Rightarrow y-f(x_0) = \frac{-1}{f'(x_0)}(x-x_0),
% \]
% % \begin{center}
% % \scalebox{1}{\input{img/calculo_diferencial_1_variable/tangente_normal}}
% % \end{center}
% \end{frame}
%
%
% % ---------------------------------------------------------------------slide----
% \begin{frame}
% \frametitle{Rectas tangente y normal a una función}
% \framesubtitle{Ejemplo}
% Dada la función $y=f(x)=x^2$, la trayectoria que dibuja la gráfica de esta función es $g(t)=(t,t^2)$ y su velocidad es
% $g'(t)=(1,2t)$, de modo que en el punto $(1,1)$, que se alcanza en el instante $t=1$, la recta tangente es
% \[
% \frac{x-1}{1} = \frac{y-1}{2} \Rightarrow y-1 = 2(x-1) \Rightarrow y = 2x-1,
% \]
% y la recta normal es
% \[
% \frac{x-1}{2} = \frac{y-1}{-1} \Rightarrow y-1 = \frac{-1}{2}(x-1) \Rightarrow y = \frac{-x}{2}+\frac{3}{2}.
% \]
% \begin{center}
% \scalebox{0.8}{\input{img/calculo_diferencial_1_variable/parabola_tangente_normal}}
% \end{center}
% \end{frame}
%
%
% % ---------------------------------------------------------------------slide----
% \begin{frame}
% \frametitle{Recta tangente a una trayectoria en el espacio}
% El concepto de recta tangente a una trayectoria en el plano real puede extenderse fácilmente a trayectorias en el espacio real $\mathbb{R}^3$.
%
% Si $f(t)=(x(t),y(t),z(t))$, $t\in I\subseteq \mathbb{R}$, es una trayectoria en el espacio real $\mathbb{R}^3$, entonces
% el móvil que recorre esta trayectoria en el instante $t=t_0$, ocupará la posición $P=(x(t_0),y(t_0),z(t_0))$ y tendrá una velocidad $\mathbf{v}=f'(t)=(x'(t),y'(t),z'(t))$, de manera que la recta tangente a $f$ en ese instante será
% \begin{align*}
% l&: (x,y,z)=(x(t_0),y(t_0),z(t_0))+t(x'(t_0),y'(t_0),z'(t_0)) =\\
% &= (x(t_0)+tx'(t_0),y(t_0)+ty'(t_0),z(t_0)+tz'(t_0)),
% \end{align*}
% cuyas ecuaciones cartesianas son
% \[
% \frac{x-x(t_0)}{x'(t_0)}=\frac{y-y(t_0)}{y'(t_0)}=\frac{z-z(t_0)}{z'(t_0)},
% \]
% siempre que $x'(t_0)\neq 0$, $y'(t_0)\neq 0$ y $z'(t_0)\neq 0$.
% \end{frame}
%
%
% % ---------------------------------------------------------------------slide----
% \begin{frame}
% \frametitle{Recta tangente a una trayectoria en el espacio}
% \framesubtitle{Ejemplo}
% Dada la trayectoria del espacio $f(t)=(\cos t, \sen t, t)$, $t\in \mathbb{R}$, en el instante $t=\pi/2$, la trayectoria
% pasará por el punto
% \[
% f(\pi/2)=(\cos(\pi/2),\sen(\pi/2),\pi/2)=(0,1,\pi/2),
% \] con una velocidad
% \[
% \mathbf{v}=f'(\pi/2)=(-\sen(\pi/2),\cos(\pi/2), 1)=(-1,0,1),
% \] y la
% tangente en ese punto es
% \[
% l:(x,y,z)=(0,1,\pi/2)+t(-1,0,1) = (-t,1,t+\pi/2).
% \]
% \begin{center}
% \scalebox{0.8}{\input{img/calculo_diferencial_1_variable/tangente_trayectoria_espacio}}
% \end{center}
% \end{frame}
%
%
% \subsection{Estudio del crecimiento de una función}
% %---------------------------------------------------------------------slide----
% \begin{frame}
% \frametitle{Estudio del crecimiento de una función}
% La principal aplicación de la derivada es el estudio del crecimiento de una función mediante el signo de la derivada.
% \begin{teorema}
% Si $f$ es una función cuya derivada existe en un intervalo $I$, entonces:
% \begin{itemize}
% \item Si $\forall x\in I\ f'(x)\geq 0$ entonces $f$ es creciente en el intervalo $I$.
% \item Si $\forall x\in I\ f'(x)\leq 0$ entonces $f$ es decreciente en el intervalo $I$.
% \end{itemize}
% \end{teorema}
% \structure{\textbf{Ejemplo}}
% La función $f(x)=x^3$ es creciente en todo $\mathbb{R}$ ya que $\forall x\in \mathbb{R}\
% f'(x)\geq 0$. \vskip .5cm
% \textbf{Observación}. \emph{Una función puede ser creciente o decreciente en un intervalo y no tener derivada.}
% \end{frame}
%
%
% %---------------------------------------------------------------------slide----
% \begin{frame}
% \frametitle{Estudio del crecimiento de una función}
% \framesubtitle{Ejemplo}
% Consideremos la función $f(x)=x^4-2x^2+1$. Su derivada $f'(x)=4x^3-4x$ está definida en todo $\mathbb{R}$ y es continua.
% \begin{center}
% \scalebox{1}{\input{img/calculo_diferencial_1_variable/estudio_crecimiento}}
% \end{center}
% \end{frame}
%
%
% \subsection{Determinación de los extremos relativos de una función}
% %---------------------------------------------------------------------slide----
% \begin{frame}
% \frametitle{Determinación de extremos relativos de una función}
% Como consecuencia del resultado anterior, la derivada también sirve para determinar los extremos relativos de una función.
% \begin{teorema}[Criterio de la primera derivada]
% Sea $f$ es una función cuya derivada existe en un intervalo $I$, y sea $x_0\in I$ tal que $f'(x_0)=0$, entonces:
% \begin{itemize}
% \item Si existe un $\delta>0$ tal que $\forall x\in(x_0-\delta,x_0)\ f'(x)>0$ y $\forall x\in(x_0,x_0+\delta)\ f'(x)<0$ entonces $f$ tiene un \emph{máximo relativo} en $x_0$.
% \item Si existe un $\delta>0$ tal que $\forall x\in(x_0-\delta,x_0)\ f'(x)<0$ y $\forall x\in(x_0,x_0+\delta)\ f'(x)>0$ entonces $f$ tiene un \emph{mínimo relativo} en $x_0$.
% \item Si existe un $\delta>0$ tal que $\forall x\in(x_0-\delta,x_0)\ f'(x)>0$ y $\forall x\in(x_0,x_0+\delta)\ f'(x)>0$ entonces $f$ tiene un \emph{punto de inflexión creciente} en $x_0$.
% \item Si existe un $\delta>0$ tal que $\forall x\in(x_0-\delta,x_0)\ f'(x)<0$ y $\forall x\in(x_0,x_0+\delta)\ f'(x)<0$ entonces $f$ tiene un \emph{punto de inflexión decreciente} en $x_0$.
% \end{itemize}
% \end{teorema}
% Los puntos donde se anula la derivada de una función se denominan \emph{puntos críticos}.
%
% %\textbf{Observación}. \emph{La anulación de la derivada es una condición necesaria pero no suficiente para la existencia de un extremo relativo en un punto.}
%
% %\structure{\textbf{Ejemplo}} La función $f(x)=x^3$ tiene derivada $f'(x)=3x^2$ y por tanto tiene un punto crítico en $x=0$, pero no tiene un extremo relativo en el 0, sino un punto de inflexión.
% \end{frame}
%
%
% %---------------------------------------------------------------------slide----
% \begin{frame}
% \frametitle{Determinación de extremos relativos de una función}
% \framesubtitle{Ejemplo}
% Consideremos de nuevo la función $f(x)=x^4-2x^2+1$.
% Su derivada $f'(x)=4x^3-4x$ está definida en todo $\mathbb{R}$ y es continua.
% \begin{center}
% \scalebox{1}{\input{img/calculo_diferencial_1_variable/estudio_extremos}}
% \end{center}
% \end{frame}
%
%
% \subsection{Estudio de la concavidad de una función}
% %---------------------------------------------------------------------slide---
% \begin{frame}
% \frametitle{Estudio de la concavidad de una función}
% La concavidad de una función puede estudiarse mediante el signo de la segunda derivada.
% \begin{teorema}[Criterio de la segunda derivada]
% Si $f$ es una función cuya segunda derivada existe en un intervalo $I$, entonces:
% \begin{itemize}
% \item Si $\forall x\in I\ f''(x)\geq 0$ entonces $f$ es cóncava en el intervalo $I$.
% \item Si $\forall x\in I\ f''(x)\leq 0$ entonces $f$ es convexa en el intervalo $I$.
% \end{itemize}
% \end{teorema}
%
% \structure{\textbf{Ejemplo}} La función $f(x)=x^2$ tiene segunda derivada $f''(x)=2>0$ y por tanto es cóncava en todo $\mathbb{R}$.
% \vskip .5cm
% \textbf{Observación}. \emph{Una función puede ser cóncava o convexa en un intervalo y no tener derivada.}
% \end{frame}
%
%
% %---------------------------------------------------------------------slide----
% \begin{frame}
% \frametitle{Estudio de la concavidad de una función}
% \framesubtitle{Ejemplo}
% Consideremos de nuevo la función $f(x)=x^4-2x^2+1$. Su segunda derivada $f''(x)=12x^2-4$ está definida en todo $\mathbb{R}$ y es continua.
% \begin{center}
% \scalebox{1}{\input{img/calculo_diferencial_1_variable/estudio_concavidad}}
% \end{center}
% \end{frame}
%
%
%
% \subsection{Polinomios de Taylor}
% %---------------------------------------------------------------------slide----
% \begin{frame}
% \frametitle{Aproximación de una función mediante un polinomio}
% Una apliación muy útil de la derivada es la aproximación de funciones mediante polinomios.
%
% Los polinomios son funciones sencillas de calcular (mediante sumas y productos), que tienen muy buenas propiedades:
% \begin{itemize}
% \item Están definidos en todos los números reales.
% \item Son funciones continuas.
% \item Son derivables hasta cualquier orden y sus derivadas son continuas.
% \end{itemize}
%
% \begin{block}{Objetivo}
% Aproximar una función $f(x)$ mediante un polinomio $p(x)$ cerca de un valor $x=x_0$.
% \end{block}
%
% \end{frame}
%
%
% %---------------------------------------------------------------------slide----
% \begin{frame}
% \frametitle{Aproximación mediante un polinomio de grado 0}
% Un polinomio de grado 0 tiene ecuación
% \[
% p(x) = c_0,
% \]
% donde $c_0$ es una constante.
%
% Como el polinomio debe valer lo que la función en el punto $x_0$, debe cumplir
% \[p(x_0) = c_0 = f(x_0).\]
%
% En consecuencia, el polinomio de grado 0 que mejor aproxima a $f$ en un entorno de $x_0$ es
% \[p(x) = f(x_0).\]
% \end{frame}
%
%
% %---------------------------------------------------------------------slide----
% \begin{frame}
% \frametitle{Aproximación mediante un polinomio de grado 0}
% \begin{center}
% \scalebox{1}{\input{img/aplicaciones_derivada/polinomio_grado_0}}
% \end{center}
% \end{frame}
%
%
% %---------------------------------------------------------------------slide----
% \begin{frame}
% \frametitle{Aproximación mediante un polinomio de grado 1}
% Un polinomio de grado 1 es una recta y tiene ecuación
% \[
% p(x) = c_0+c_1x,
% \]
% aunque también puede escribirse
% \[
% p(x) = c_0+c_1(x-x_0).
% \]
%
% De entre todos los polinomios de grado 1, el que mejor aproxima a $f$ en entorno de $x_0$ será el que cumpla las dos condiciones siguientes:
% \begin{itemize}
% \item[\structure{1-}] $p$ y $f$ valen lo mismo en $x_0$: $p(x_0) = f(x_0)$,
% \item[\structure{2-}] $p$ y $f$ tienen la misma tasa de crecimiento en $a$: $p'(x_0) = f'(x_0)$.
% \end{itemize}
% Esta última condición nos asegura que en un entorno de $x_0$, $p$ y $f$ tienen aproximadamente la misma tendencia de crecimiento, pero requiere que la función $f$ sea derivable en $x_0$.
% \end{frame}
%
%
% %---------------------------------------------------------------------slide----
% \begin{frame}
% \frametitle{La recta tangente: Mejor aproximación de grado 1}
% Imponiendo las condiciones anteriores tenemos
% \begin{itemize}
% \item[\structure{1-}] $p(x)=c_0+c_1(x-x_0) \Rightarrow p(x_0)=c_0+c_1(x_0-x_0)=c_0=f(x_0)$,
% \item[\structure{2-}] $p'(x)=c_1 \Rightarrow p'(x_0)=c_1=f'(x_0)$.
% \end{itemize}
%
% Así pues, el polinomio de grado 1 que mejor aproxima a $f$ en un entorno de $x_0$ es
% \[
% p(x) = f(x_0)+f '(x_0)(x-x_0),
% \]
% que resulta ser la recta tangente a $f$ en el punto $(x_0,f(x_0))$.
% \end{frame}
%
%
% %---------------------------------------------------------------------slide----
% \begin{frame}
% \frametitle{Aproximación mediante un polinomio de grado 1}
% \begin{center}
% \scalebox{1}{\input{img/aplicaciones_derivada/polinomio_grado_1}}
% \end{center}
% \end{frame}
%
%
% %---------------------------------------------------------------------slide----
% \begin{frame}
% \frametitle{Aproximación mediante un polinomio de grado 2}
% Un polinomio de grado 2 es una parábola y tiene ecuación
% \[
% p(x) = c_0+c_1x+c_2x^2,
% \]
% aunque también puede escribirse
% \[
% p(x) = c_0+c_1(x-x_0)+c_2(x-x_0)^2.
% \]
%
% De entre todos los polinomio de grado 2, el que mejor aproxima a $f$ en entorno de $x_0$ será el que cumpla las tres condiciones siguientes:
% \begin{itemize}
% \item[\structure{1-}] $p$ y $f$ valen lo mismo en $x_0$: $p(x_0) = f(x_0)$,
% \item[\structure{2-}] $p$ y $f$ tienen la misma tasa de crecimiento en $x_0$: $p'(x_0) = f'(x_0)$.
% \item[\structure{3-}] $p$ y $f$ tienen la misma curvatura en $x_0$: $p''(x_0)=f''(x_0)$.
% \end{itemize}
% Esta última condición requiere que la función $f$ sea dos veces derivable en $x_0$.
% \end{frame}
%
%
% %---------------------------------------------------------------------slide----
% \begin{frame}
% \frametitle{Mejor polinomio de grado 2}
% Imponiendo las condiciones anteriores tenemos
% \begin{itemize}
% \item[\structure{1-}] $p(x)=c_0+c_1(x-x_0)+c_2(x-x_0)^2 \Rightarrow p(x_0)=c_0=f(x_0)$,
% \item[\structure{2-}] $p'(x)=c_1+2c_2(x-x_0) \Rightarrow p'(x_0)=c_1+2c_2(x_0-x_0)=c_1=f'(x_0)$,
% \item[\structure{3-}] $p''(x)=2c_2 \Rightarrow p''(x_0)=2c_2=f''(x_0) \Rightarrow c_2=\frac{f''(x_0)}{2}$.
% \end{itemize}
%
% Así pues, el polinomio de grado 2 que mejor aproxima a $f$ en un entorno de $x_0$ es
% \[
% p(x) = f(x_0)+f'(x_0)(x-x_0)+\frac{f''(x_0)}{2}(x-x_0)^2.
% \]
% \end{frame}
%
%
% %---------------------------------------------------------------------slide----
% \begin{frame}
% \frametitle{Aproximación mediante un polinomio de grado 2}
% \begin{center}
% \scalebox{1}{\input{img/aplicaciones_derivada/polinomio_grado_2}}
% \end{center}
% \end{frame}
%
%
% %---------------------------------------------------------------------slide----
% \begin{frame}
% \frametitle{Aproximación mediante un polinomio de grado $n$}
% Un polinomio de grado $n$ tiene ecuación
% \[p(x) = c_0+c_1x+c_2x^2+\cdots +c_nx^n,\]
% aunque también puede escribirse
% \[p(x) = c_0+c_1(x-x_0)+c_2(x-x_0)^2+\cdots +c_n(x-x_0)^n.\]
%
% De entre todos los polinomio de grado $n$, el que mejor aproxima a $f$ en entorno de $x_0$ será el que cumpla las $n+1$ condiciones siguientes:
% \begin{itemize}
% \item[\structure{1-}] $p(x_0) = f(x_0)$,
% \item[\structure{2-}] $p'(x_0) = f'(x_0)$,
% \item[\structure{3-}] $p''(x_0)=f''(x_0)$,
% \item[] $\cdots$
% \item[\structure{n+1-}] $p^{(n}(x_0)=f^{(n}(x_0)$.
% \end{itemize}
% \alert{Obsérvese que para que se cumplan estas condiciones es necesario que $f$ sea $n$ veces derivable en $x_0$.}
% \end{frame}
%
%
% %---------------------------------------------------------------------slide----
% \begin{frame}
% \frametitle{Cálculo de los coeficientes del polinomio de grado $n$}
% Las sucesivas derivadas de $p$ valen
% \begin{align*}
% p(x) &= c_0+c_1(x-x_0)+c_2(x-x_0)^2+\cdots +c_n(x-x_0)^n,\\
% p'(x)& = c_1+2c_2(x-x_0)+\cdots +nc_n(x-x_0)^{n-1},\\
% p''(x)& = 2c_2+\cdots +n(n-1)c_n(x-x_0)^{n-2},\\
% \vdots\ \
% \\
% p^{(n}(x)&= n(n-1)(n-2)\cdots 1 c_n=n!c_n.
% \end{align*}
%
% Imponiendo las condiciones anteriores se tiene
% \begin{itemize}
% \item[\structure{1-}] $p(x_0) = c_0+c_1(x_0-x_0)+c_2(x_0-x_0)^2+\cdots +c_n(x_0-x_0)^n=c_0=f(x_0)$,
% \item[\structure{2-}] $p'(x_0) = c_1+2c_2(x_0-x_0)+\cdots +nc_n(x_0-x_0)^{n-1}=c_1=f'(x_0)$,
% \item[\structure{3-}] $p''(x_0) = 2c_2+\cdots +n(n-1)c_n(x_0-x_0)^{n-2}=2c_2=f''(x_0)\Rightarrow c_2=f''(x_0)/2$,
% \item[] $\cdots$
% \item[\structure{n+1-}] $p^{(n}(x_0)=n!c_n=f^{(n}(x_0)=c_n=\frac{f^{(n}(x_0)}{n!}$.
% \end{itemize}
% \end{frame}
%
%
% %---------------------------------------------------------------------slide----
% \begin{frame}
% \frametitle{Polinomio de Taylor de orden $n$}
% \begin{definicion}[Polinomio de Taylor de orden $n$ para $f$ en el punto $a$]
% Dada una función $f$, $n$ veces derivable en $x=x_0$, se define el \emph{polinomio de Taylor} de orden $n$ para $f$ en $x_0$ como
% \begin{align*}
% p_{f,x_0}^n(x)&=f(x_0)+f'(x_0)(x-x_0)+\frac{f''(x_0)}{2}(x-x_0)^2+\cdots +\frac{f^{(n}(x_0)}{n!}(x-x_0)^n = \\ &=\sum_{i=0}^{n}\frac{f^{(i}(x_0)}{i!}(x-x_0)^i,
% \end{align*}
% o bien, escribiendo $x=x_0+h$
% \[
% p_f^n(x_0+h)&=f(x_0)+f'(x_0)h+\frac{f''(x_0)}{2}h^2+\cdots +\frac{f^{(n}(x_0)}{n!}h^n =\sum_{i=0}^{n}\frac{f^{(i}(x_0)}{i!}h^i,
% \]
% \end{definicion}
%
% El polinomio de Taylor de orden $n$ para $f$ en $x_0$ es el polinomio de orden $n$ que mejor aproxima a $f$ alrededor de $x_0$, ya que es el único que cumple las $n+1$ condiciones anteriores.
% \end{frame}
%
%
% %---------------------------------------------------------------------slide----
% \begin{frame}
% \frametitle{Cálculo del polinomio de Taylor}
% \framesubtitle{Ejemplo}
% Vamos a aproximar la función $f(x)=\log x$ en un entorno del punto $1$ mediante un polinomio de grado $3$.
%
% La ecuación del polinomio de Taylor de orden $3$ para $f$ en el punto $1$ es
% \[
% p_{f,1}^3(x)=f(1)+f'(1)(x-1)+\frac{f''(1)}{2}(x-1)^2+\frac{f'''(1)}{3!}(x-1)^3.
% \]
% Calculamos las tres primeras derivadas de $f$ en $1$:
% \[
% \begin{array}{lll}
% f(x)=\log x & \quad & f(1)=\log 1 =0,\\
% f'(x)=1/x & & f'(1)=1/1=1,\\
% f''(x)=-1/x^2 & & f''(1)=-1/1^2=-1,\\
% f'''(x)=2/x^3 & & f'''(1)=2/1^3=2.
% \end{array}
% \]
% Sustituyendo en la ecuación del polinomio se tiene
% \[
% p_{f,1}^3(x)=0+1(x-1)+\frac{-1}{2}(x-1)^2+\frac{2}{3!}(x-1)^3= \frac{2}{3}x^3-\frac{3}{2}x^2+3x-\frac{11}{6}.
% \]
% \end{frame}
%
%
% %---------------------------------------------------------------------slide----
% \begin{frame}
% \frametitle{Polinomios de Taylor para la función logaritmo}
% \begin{center}
% \scalebox{1}{\input{img/aplicaciones_derivada/polinomio_taylor_logaritmo}}
% \end{center}
% \end{frame}
%
%
% %---------------------------------------------------------------------slide----
% \begin{frame}
% \frametitle{Polinomio de Mc Laurin de orden $n$}
% La ecuación del polinomio de Taylor se simplifica cuando el punto en torno al cual queremos aproximar es el $0$.
% \begin{definicion}[Polinomio de Mc Laurin de orden $n$ para $f$]
% Dada una función $f$, $n$ veces derivable en $0$, se define el \emph{polinomio de Mc Laurin} de orden $n$ para $f$ como
% \begin{align*}
% p_{f,0}^n(x)&=f(0)+f'(0)x+\frac{f''(0)}{2}x^2+\cdots +\frac{f^{(n}(0)}{n!}x^n = \\ &=\sum_{i=0}^{n}\frac{f^{(i}(0)}{i!}x^i.
% \end{align*}
% \end{definicion}
% \end{frame}
%
%
% %---------------------------------------------------------------------slide----
% \begin{frame}
% \frametitle{Cálculo del polinomio de Mc Laurin}
% \framesubtitle{Ejemplo}
% Vamos a aproximar la función $f(x)=\sen x$ en un entorno del punto $0$ mediante un polinomio de grado $3$.
%
% La ecuación del polinomio de Mc Laurin de orden $3$ para $f$ es
% \[
% p_{f,0}^3(x)=f(0)+f'(0)x+\frac{f''(0)}{2}x^2+\frac{f'''(0)}{3!}x^3.
% \]
% Calculamos las tres primeras derivadas de $f$ en $0$:
% \[
% \begin{array}{lll}
% f(x)=\sen x & \quad & f(0)=\sen 0 =0,\\
% f'(x)=\cos x & & f'(0)=\cos 0=1,\\
% f''(x)=-\sen x & & f''(0)=-\sen 0=0,\\
% f'''(x)=-\cos x & & f'''(0)=-\cos 0=-1.
% \end{array}
% \]
% Sustituyendo en la ecuación del polinomio obtenemos
% \[
% p_{f,0}^3(x)=0+1\cdot x+\frac{0}{2}x^2+\frac{-1}{3!}x^3= x-\frac{x^3}{6}.
% \]
% \end{frame}
%
%
% %---------------------------------------------------------------------slide----
% \begin{frame}
% \frametitle{Polinomios de Mc Laurin para la función seno}
% \begin{center}
% \scalebox{1}{\input{img/aplicaciones_derivada/polinomio_mclaurin_seno}}
% \end{center}
% \end{frame}
%
%
% %---------------------------------------------------------------------slide----
% \begin{frame}
% \frametitle{Polinomios de Mc Laurin de funciones elementales}
% \[
% \renewcommand{\arraystretch}{2.5}
% \begin{array}{|c|c|}
% \hline
% f(x) & p_{f,0}^n(x) \\
% \hline\hline
% \sen x & \displaystyle x-\frac{x^3}{3!}+\frac{x^5}{5!}-\cdots +(-1)^k\frac{x^{2k-1}}{(2k-1)!} \mbox{ si $n=2k$ o $n=2k-1$}\\
% \hline
% \cos x &  \displaystyle 1-\frac{x^2}{2!}+\frac{x^4}{4!}-\cdots +(-1)^k\frac{x^{2k}}{(2k)!} \mbox{ si $n=2k$ o $n=2k+1$}\\
% \hline
% \arctg x &  \displaystyle x-\frac{x^3}{3}+\frac{x^5}{5}-\cdots +(-1)^k\frac{x^{2k-1}}{(2k-1)} \mbox{ si $n=2k$ o $n=2k-1$}\\
% \hline
% e^x & \displaystyle 1+x+\frac{x^2}{2!}+\frac{x^3}{3!}+\cdots + \frac{x^n}{n!}\\
% \hline
% \log(1+x) & \displaystyle x-\frac{x^2}{2}+\frac{x^3}{3}-\cdots +(-1)^{n-1}\frac{x^n}{n}\\
% \hline
% \end{array}
% \]
% \end{frame}
%
%
% %---------------------------------------------------------------------slide----
% \begin{frame}
% \frametitle{Resto de Taylor}
% Los polinomios de Taylor permiten calcular el valor aproximado de una función cerca de un valor $x_0$, pero siempre se comete un error en dicha aproximación.
% \begin{definicion}[Resto de Taylor]
% Si $f$ es una función para la que existe el su polinomio de Taylor de orden $n$ en $x_0$, $p_{f,x_0}^n$, entonces se define el \emph{resto de Taylor} de orden $n$ para $f$ en $x_0$ como
% \[
% r_{f,x_0}^n(x)=f(x)-p_{f,x_0}^n(x).
% \]
% \end{definicion}
%
% El resto mide el error cometido al aproximar $f(x)$ mediante $p_{f,x_0}^n(x)$ y permite expresar la función $f$ como la suma de un polinomio de Taylor más su resto correspondiente:
% \[
% f(x)=p_{f,x_0}^n(x) + r_{f,x_0}^n(x).
% \]
% Esta expresión se conoce como \emph{fórmula de Taylor} de orden $n$ para $f$ en $x_0$. Se pude demostrar, además, que
% \[
% \lim_{h\rightarrow 0}\frac{r_{f,x_0}^n(x_0+h)}{h^n}=0,
% \]
% lo cual indica que el resto $r_{f,x_0}^n(x_0+h)$ es mucho menor que $h^n$.
% \end{frame}
%




%
% \subsection{El concepto de diferencial}
% %---------------------------------------------------------------------slide----
% \begin{frame}
% \frametitle{El concepto de diferencial}
% \begin{definicion}[Diferencial de una función en un punto]
% Dada una función $f$, se llama \emph{diferencial} de $f$ en un punto $a$, al la función
% \[
% \begin{array}{rccc}
% dy=df(a): & \mathbb{R} & \longrightarrow & \mathbb{R} \\
% & \Delta x & \longrightarrow & f'(a)\Delta x
% \end{array}
% \]
% \end{definicion}
%
% Cuando $f$ es la función identidad $y=x$,  entonces $f'(a)=1$, y se cumple que
% \[ dx=dy=f'(a)\Delta x=\Delta x,\]
% de modo que también podemos definir el diferencial como
% \[dy=df(a)=f'(a)dx.\]
%
% De aquí se deduce otra forma de escribir la derivada de $f$ en $a$
% \[f'(a)=\frac{dy}{dx}=\frac{df(a)}{dx}.\]
% \end{frame}
%
%
% %---------------------------------------------------------------------slide----
% \begin{frame}
% \frametitle{Aproximación de una función mediante su diferencial}
% El diferencial de una función $f$ en un punto $a$, permite aproximar la variación de $f$ cerca de $a$.
%
% \begin{center}
% \scalebox{1}{\input{img/calculo_diferencial_1_variable/diferencial}}
% \end{center}
% \end{frame}
%
%
% %---------------------------------------------------------------------slide----
% \begin{frame}
% \frametitle{Aproximación de una función mediante su diferencial: Ejemplo}
% Consideremos otra vez la función $y=x^2$ que mide el área de un cuadrado de chapa metálica de lado $x$.
%
% Si el lado del cuadrado es $a$, y sometemos la chapa a un proceso de calentamiento que aumenta el lado del cuadrado, ¿cuál  será aproximadamente la variación que habrá experimentado el área, cuando el lado aumente una cantidad $dx$?
% \begin{columns}
% \begin{column}{0.3\textwidth}
% \begin{align*}
% \Delta y &= f(a+dx)-f(a)=(a+dx)^2-a^2=\\
% &= a^2+2adx+dx^2-a^2=2adx+dx^2,\\
% \only<2->{dy &= f'(a)dx= 2adx.}
% \end{align*}
% \uncover<3->{Además, \[\lim_{dx\rightarrow 0}\Delta y-dy=\lim_{dx\rightarrow 0}dx^2=0.\]}
% \end{column}
% \begin{column}{0.3\textwidth}
% \begin{center}
% \scalebox{1}{\input{img/calculo_diferencial_1_variable/diferencial_area_cuadrado}}
% \end{center}
% \end{column}
% \end{columns}
% \end{frame}


% %---------------------------------------------------------------------slide----
% \begin{frame}
% \frametitle{Propiedades del diferencial}
% Si $y=c$, es una función constante, entonces $dy=0$.
% Si $y=x$, es la función identidad, entonces  $dy=dx$.
%
% Si $u=f(x)$ y $v=g(x)$ son dos funciones diferenciables, entonces
% \begin{itemize}
% \item $d(u+v)=d(u)+d(v)$
% \item $d(u-v)=d(u)-d(v)$
% \item $d(u\cdot v)=d(u)\cdot v+ u\cdot d(v)$
% \item $d\left(\dfrac{u}{v}\right)=\dfrac{du\cdot v-u\cdot dv}{v^2}$
% \end{itemize}
% \end{frame}





% \subsection{Derivada de una función implícita}
% %---------------------------------------------------------------------slide----
% \begin{frame}
% \frametitle{Derivada de una función implícita}
% Si $F(x,y)=0$ es una función implícita entonces
% \[
% dF(x,y)=d0=0.
% \]
%
% Si $F(x,y)=0$ es una función implícita en la que $y$ depende de $x$, entonces podemos calcular la derivada de $y$ a partir del diferencial
% \[
% \frac{dF(x,y)}{dx}=\frac{d0}{dx}=0.
% \]
% \structure{\textbf{Ejemplo}}. Consideremos la función implícita de la circunferencia de radio 1, $x^2+y^2=1$. Entonces su diferencial es
% \[
% d(x^2+y^2)=d1=0 \Leftrightarrow d(x^2)+d(y^2)=2x\;dx+2y\;dy=0.
% \]
% A partir de aquí podemos calcular fácilmente la derivada de $y$:
% \[
% \frac{d(x^2+y^2)}{dx}= \frac{2x\;dx+2y\;dy}{dx}=2x\frac{dx}{dx}+2y\frac{dy}{dx}= 2x+2y\frac{dy}{dx}=0 \Leftrightarrow \frac{dy}{dx}=\frac{-x}{y}.
% \]
%
% \end{frame}
%
%
% \subsection{Derivada de una función paramétrica}
% %---------------------------------------------------------------------slide----
% \begin{frame}
% \frametitle{Derivada de una función parametrica}
%
% Dada una función paramétrica
% \[
% \left\{%
% \begin{array}{l}
% x=f(t) \\
% y=g(t)
% \end{array}%
% \right.
% \]
% podemos calcular su derivada a partir de las derivadas de $f$ y $g$:
% \[\frac{dy}{dx}=\frac{g'(t)\,dt}{f'(t)\,dt}=\frac{g'(t)}{f'(t)}.\]
%
% \structure{\textbf{Ejemplo}}. Consideremos la elipse
% \[
% \left\{%
% \begin{array}{l}
% x=2\sen t \\
% y=\cos t
% \end{array}%
% \right.
% \]
% Entonces
% \[
% \frac{dy}{dx}=\frac{-\sen t\; dt}{2 \cos t\; dt}=\frac{-1}{2}\tg t.
% \]
% \end{frame}
