% !TEX root = ../calculus_manual.tex
\section{Differential calculus with one real variable}
\mode<presentation>{
%---------------------------------------------------------------------slide----
\begin{frame}
\frametitle{Differential calculus with one variable}
\tableofcontents[sectionstyle=show/hide,hideothersubsections]
\end{frame}
}


\subsection{Concept of derivative}
%---------------------------------------------------------------------slide----
\begin{frame}
\frametitle{Increment}
\begin{definition}[Increment of a variable]
An \emph{increment} of a variable $x$ is a change in the value of the variable and is denoted $\Delta x$.
The increment of a variable $x$ along an interval $[a,b]$ is
\[
\Delta x = b-a.
\]
\end{definition}

\begin{definition}[Increment of a function]
The \emph{increment} of a function $y=f(x)$ along an interval $[a,b]\subseteq Dom(f)$ is
\[
\Delta y = f(b)-f(a).
\]
\end{definition}

\textbf{Example} The increment of $x$ along the interval $[2,5]$ is $\Delta x=5-2=3$ and the increment of the function $y=x^2$ along the same interval is $\Delta y=5^2-2^2=21$.
\end{frame}

\begin{frame}
\frametitle{Average rate of change}
The study of a function $y=f(x)$ requires to understand how the function changes, that is, how changes the dependent variable $y$ when we change the independent variable $x$.

\begin{definition}[Average rate of change]
The  \emph{average rate of change} of a function $y=f(x)$ in an interval $[a,a+\Delta x]\subseteq Dom(f)$, is the quotient between the increment of $y$ and the increment of $x$ in that interval, and is denoted
\[
\mbox{ARC}\;f[a,a+\Delta x]=\frac{\Delta y}{\Delta x}=\frac{f(a+\Delta x)-f(a)}{\Delta x}.
\]
\end{definition}
\end{frame}

%---------------------------------------------------------------------slide----
\begin{frame}
\frametitle{Average rate of change}
\framesubtitle{Example of the area of a square}
Let $y=x^2$ be the function that measures the area of a metallic square of side $x$.

If at any given time the side of the square is $a$, and we heat the square uniformly increasing the side by dilatation a quantity $\Delta x$, how much will increase the area of the square?
\begin{columns}
\begin{column}{0.3\textwidth}
\begin{align*}
\Delta y &= f(a+\Delta x)-f(a)=(a+\Delta x)^2-a^2=\\
&= a^2+2a\Delta x+\Delta x^2-a^2=2a\Delta x+\Delta x^2.
\end{align*}
\end{column}
\begin{column}{0.3\textwidth}
\begin{center}
\tikzsetnextfilename{derivatives_1_variable/square_area_variation}
\mode<article>{\resizebox{0.3\textwidth}{!}{\begin{tikzpicture}
% \draw (0,0) node[below] {$a$} rectangle (4,4) node[midway] {$a^2$};
% \draw [fill=orange] (0,4) rectangle (4,5) node[midway] {$a\Delta x$};
% \draw [fill=orange] (4,0) rectangle (5,4) node[midway] {$a\Delta x$};
% \draw [fill=orange] (4,4) rectangle (5,5) node[midway] {$\Delta x^2$};
\node (A) [rectangle, draw, minimum width=4cm, minimum height=4cm, label={[anchor=north]south:$a$}] at (0,0) {$a^2$};
\node (B) [anchor=south, rectangle, draw, fill=color4!50, minimum width=4cm, minimum height=1cm] at (A.north) {$a\Delta x$};
\node (C) [anchor=west, rectangle, draw, fill=color4!50, minimum width=1cm, minimum height=4cm, label={[anchor=north]south:$\Delta x$}] at (A.east) {$a\Delta x$};
\node (D) [anchor=south, rectangle, draw, fill=color4!50, minimum width=1cm, minimum height=1cm] at (C.north) {$\Delta x^2$};
\end{tikzpicture}
}}
\mode<presentation>{\resizebox{0.9\textwidth}{!}{\begin{tikzpicture}
% \draw (0,0) node[below] {$a$} rectangle (4,4) node[midway] {$a^2$};
% \draw [fill=orange] (0,4) rectangle (4,5) node[midway] {$a\Delta x$};
% \draw [fill=orange] (4,0) rectangle (5,4) node[midway] {$a\Delta x$};
% \draw [fill=orange] (4,4) rectangle (5,5) node[midway] {$\Delta x^2$};
\node (A) [rectangle, draw, minimum width=4cm, minimum height=4cm, label={[anchor=north]south:$a$}] at (0,0) {$a^2$};
\node (B) [anchor=south, rectangle, draw, fill=color4!50, minimum width=4cm, minimum height=1cm] at (A.north) {$a\Delta x$};
\node (C) [anchor=west, rectangle, draw, fill=color4!50, minimum width=1cm, minimum height=4cm, label={[anchor=north]south:$\Delta x$}] at (A.east) {$a\Delta x$};
\node (D) [anchor=south, rectangle, draw, fill=color4!50, minimum width=1cm, minimum height=1cm] at (C.north) {$\Delta x^2$};
\end{tikzpicture}
}}
\end{center}
\end{column}
\end{columns}
What is the average rate of change in the interval $[a,a+\Delta x]$?
\[
\mbox{ARC}\;f[a,a+\Delta x]=\frac{\Delta y}{\Delta x}=\frac{2a\Delta x+\Delta x^2}{\Delta x}=2a+\Delta x.
\]
\end{frame}


%---------------------------------------------------------------------slide----
\begin{frame}
\frametitle{Geometric interpretation of the average rate of change}
The average rate of change of a function $y=f(x)$ in an interval $[a,a+\Delta x]$ is the slope of the \emph{secant} line of $f$ through the points $(a,f(a))$ and $(a+\Delta x,f(a+\Delta x))$.
\begin{center}
\tikzsetnextfilename{derivatives_1_variable/secant_line}
% Author: Alfredo Sánchez Alberca (asalber@ceu.es)
\begin{tikzpicture}
  \begin{axis}[
    gen2dfun, 
    xmin=0, xmax=4.5,
    ymin=0, ymax=4,
    axis equal=true,  
    xtick={1,3},
    xticklabels={$a$,$a+\Delta x$},
    ytick={1,2.5},
    yticklabels={$f(a)$,$f(a+\Delta x)$}, 
%   ticks=none,
% 	extra y ticks={0.5,0.3},
% 	extra y tick labels={$f(a)$, $f(a+\Delta x)$},
% 	extra x ticks={0.5, 3},
% 	extra x tick labels={$a$,$a+\Delta x$},
    height=5cm,
    ]
    \addplot+[domain=0.2:3.5, smooth, name path=F] {2^(x-2)+0.5} node[anchor=south] {$f(x)$};
    \addplot+[domain=0.2:3.5, smooth, name path=S] {1+1.5/2*(x-1)} node[anchor=west] {Secant};
    \fill [name intersections={of=F and S, by={A,B}}]
      (A) circle (1.2pt)
      (B) circle (1.2pt);
    \coordinate (O);
    \draw[gray, dotted] (A) -- (A|-O);
    \draw[gray, dotted] (A) -- (A-|O);
    \draw[gray, dotted] (B) -- (B|-O);
    \draw[gray, dotted] (B) -- (B-|O);
    \draw (A) -- (A-|B) node [midway, below] {$\Delta x$};
    \draw (B) -- (B|-A) node [midway, right] {$\Delta y$};
  \end{axis}
\end{tikzpicture}

\end{center}
\end{frame} 


%---------------------------------------------------------------------slide----
\begin{frame}
\frametitle{Instantaneous rate of change}
Often is interesting to study the rate of change of a function, not in an interval, but in a point.

Knowing the tendency of change of a function in an instant can be used to predict the value of the function in next instants.

\begin{definition}[Instantaneous rate of change and derivative]
The \emph{instantaneous rate of change} of a function $f$ in a point $a$, is the limit of the average rate of change of $f$ in the interval $[a,a+\Delta x]$, when $\Delta x$ tends to 0, and is denoted
\[
\textrm{IRC}\;f (a)=\lim_{\Delta x\rightarrow 0} \textrm{ARC}\; f[a,a+\Delta x]=\lim_{\Delta x\rightarrow 0}\frac{\Delta y}{\Delta x}=\lim_{\Delta x\rightarrow 0}\frac{f(a+\Delta x)-f(a)}{\Delta x}
\]
When this limit exists, the function $f$ is said to be \emph{derivable} or \emph{differentiable} at the point $a$, and its value is called \emph{derivative} of $f$ at $a$, and denoted $f'(a)$ (Lagrange's notation) or $\frac{df}{dx}(a)$ (Leibniz's notation).
\end{definition}
\end{frame}


%---------------------------------------------------------------------slide----
\begin{frame}
\frametitle{Instantaneous rate of change}
\framesubtitle{Example of the area of a square}
Let's take again the function $y=x^2$ that measures the area of a metallic square of side $x$.

If at any given time the side of the square is $a$, and we heat the square uniformly increasing the side, what is the tendency of change of the area in that moment?
\begin{align*}
\textrm{IRC}\;f(a)&=\lim_{\Delta x\rightarrow 0}\frac{\Delta y}{\Delta x}=\lim_{\Delta x\rightarrow 0}\frac{f(a+\Delta x)-f(a)}{\Delta x} =\\
&=\lim_{\Delta x\rightarrow 0}\frac{2a\Delta x+\Delta x^2}{\Delta x}=\lim_{\Delta x\rightarrow 0} 2a+\Delta x= 2a.
\end{align*}
Thus,
\[
f'(a)=\frac{df}{dx}(a)=2a,
\]
indicating that the area of the square tend to increase the double of the side.
\end{frame}


%---------------------------------------------------------------------slide----
\begin{frame}
\frametitle{Interpretation of the derivative}
The derivative of a function $f'(a)$ shows the growth rate of $f$ at point $a$:
\begin{itemize}
\item $f'(a)>0$ indicates an increasing tendency ($y$ increases as $x$ increases).
\item $f'(a)<0$ indicates a decreasing tendency ($y$ decreases as $x$ increases).
\end{itemize}

\structure{\textbf{Example}} A derivative $f'(a)=3$ indicates that $y$ tends to increase triple of $x$ at point $a$. 
 A derivative $f'(a)=-0.5$ indicates that $y$ tends to decrease half of $x$ at point $a$. 
\end{frame}


%---------------------------------------------------------------------slide----
\begin{frame}
\frametitle{Geometric interpretation of the derivative}
\mode<article>{We have seen that the average rate of change of a function $y=f(x)$ in an interval $[a,a+\Delta x]$ is the slope of the \emph{secant} line, but when $\Delta x$ tends to $0$, the secant line becomes the tangent line.}

The instantaneous rate of change or derivative of a function $y=f(x)$ at $x=a$ is the slope of the \emph{tangent line} to $f$ at point $(a,f(a))$. 
Thus, the equation of the tangent line to $f$ at the point $(a,f(a))$ is
\[
y-f(a) = f'(a)(x-a) \Leftrightarrow y = f(a)+f'(a)(x-a)
\]
\begin{center}
\tikzsetnextfilename{derivatives_1_variable/tangent_line}
% Author: Alfredo Sánchez Alberca (asalber@ceu.es)
\begin{tikzpicture}
  \begin{axis}[
    gen2dfun, 
    xmin=0, xmax=4.5,
    ymin=0, ymax=4,
    axis equal=true,  
    xtick={1},
    xticklabels={$a$},
    ytick={1},
    yticklabels={$f(a)$}, 
    height=5cm,
    ]
    \addplot+[domain=0.2:3.5, smooth, name path=F] {2^(x-2)+0.5} node[anchor=south] {$f(x)$};
    \addplot+[domain=0.2:3.5, smooth, name path=T] {1+ln(2)/2*(x-1)} node[anchor=north west, pos=0.8] {Tangent} node [anchor=north, pos=0.8] {$y=f(a)+f'(a)(x-a)$};
    \coordinate (O);
    \coordinate (A) at (1,1);
    \fill (A) circle (1.2pt);
    \draw[gray, dotted] (A) -- (A|-O);
    \draw[gray, dotted] (A) -- (A-|O);
  \end{axis}
\end{tikzpicture}

\end{center}
\end{frame}


%---------------------------------------------------------------------slide----
\begin{frame}
\frametitle{Approximating a function with the derivative}
The tangent line to a function $f(x)$ at $x=a$ can be used to approximate $f$ in a neighbourhood of $a$.
\begin{center}
\tikzsetnextfilename{derivatives_1_variable/tangent_line_approximation}
% Author: Alfredo Sánchez Alberca (asalber@ceu.es)
\begin{tikzpicture}
  \begin{axis}[
    gen2dfun, 
    xmin=0, xmax=4.5,
    ymin=0, ymax=3.5,
    xtick={1,3},
    xticklabels={$a$, $a+\Delta x$},
    ytick={1,2.5,1.693147},
    yticklabels={$f(a)$,$f(a+\Delta x)$, $f(a)+f'(a)\Delta x$},
    clip=false,
    height=4cm,
    ]
    \addplot+[domain=0.2:3.5, smooth] {2^(x-2)+0.5} node[anchor=south] {$f(x)$};
    \addplot+[domain=0.2:3.5, smooth] {1+ln(2)/2*(x-1)} node[anchor=south west] {Tangent} node [anchor=west] {$y=f(a)+f'(a)(x-a)$};
    \coordinate (O);
    \coordinate (A) at (1,1);
    \coordinate (B) at (3,2.5);
    \coordinate (C) at (3,1.693147);
    \fill (A) circle (1.2pt);
    \draw[gray, dotted] (A) -- (A|-O);
    \draw[gray, dotted] (A) -- (A-|O);
    \draw[gray, dotted] (B) -- (B|-O);
    \draw[gray, dotted] (B) -- (B-|O);
    \draw[gray, dotted] (C) -- (C-|O);
    \draw (C) -- (C|-A) node[midway, anchor=west] {$f'(a)\Delta x$};
    \draw (A) -- (A-|C) node[midway, anchor=north] {$\Delta x$};
    \draw[color3] (B) -- (C) node[midway, anchor=east] {Error};
  \end{axis}
\end{tikzpicture}

\end{center}
Thus, the increment of a function $f(x)$ in an interval $[a,a+\Delta x]$ can be approximated multiplying the derivative of $f$ at $a$ by the increment of $x$
\[
\Delta y \approx f'(a)\Delta x
\]
\end{frame}


%---------------------------------------------------------------------slide----
\begin{frame}
\frametitle{Approximating a function with the derivative}
\framesubtitle{Example of the area of a square}
In the previous example of the function $y=x^2$ that measures the area of a metallic square of side $x$, if the side of the square is $a$ and we increment it a quantity $\Delta x$, then the increment on the area will be approximately
\[
\Delta y \approx f'(a)\Delta x = 2a\Delta x.
\]
In the figure below we can see that the error of this approximation is $\Delta x^2$, and is smaller than $\Delta x$ when $\Delta x$ tends to 0. 
\begin{center}
\tikzsetnextfilename{derivatives_1_variable/square_area_variation_approximation}
\mode<article>{\resizebox{0.3\textwidth}{!}{\begin{tikzpicture}
% \draw (0,0) node[below] {$a$} rectangle (4,4) node[midway] {$a^2$};
% \draw [fill=orange] (0,4) rectangle (4,5) node[midway] {$a\Delta x$};
% \draw [fill=orange] (4,0) rectangle (5,4) node[midway] {$a\Delta x$};
% \draw [fill=orange] (4,4) rectangle (5,5) node[midway] {$\Delta x^2$};
\node (A) [rectangle, draw, minimum width=4cm, minimum height=4cm, label={[anchor=north]south:$a$}] at (0,0) {$a^2$};
\node (B) [anchor=south, rectangle, draw, fill=color3!50, minimum width=4cm, minimum height=1cm] at (A.north) {$a\Delta x$};
\node (C) [anchor=west, rectangle, draw, fill=color3!50, minimum width=1cm, minimum height=4cm, label={[anchor=north]south:$\Delta x$}] at (A.east) {$a\Delta x$};
\node (D) [anchor=south, rectangle, draw, fill=color2!50, minimum width=1cm, minimum height=1cm] at (C.north) {$\Delta x^2$};
\end{tikzpicture}
}}
\mode<presentation>{\resizebox{0.35\textwidth}{!}{\begin{tikzpicture}
% \draw (0,0) node[below] {$a$} rectangle (4,4) node[midway] {$a^2$};
% \draw [fill=orange] (0,4) rectangle (4,5) node[midway] {$a\Delta x$};
% \draw [fill=orange] (4,0) rectangle (5,4) node[midway] {$a\Delta x$};
% \draw [fill=orange] (4,4) rectangle (5,5) node[midway] {$\Delta x^2$};
\node (A) [rectangle, draw, minimum width=4cm, minimum height=4cm, label={[anchor=north]south:$a$}] at (0,0) {$a^2$};
\node (B) [anchor=south, rectangle, draw, fill=color3!50, minimum width=4cm, minimum height=1cm] at (A.north) {$a\Delta x$};
\node (C) [anchor=west, rectangle, draw, fill=color3!50, minimum width=1cm, minimum height=4cm, label={[anchor=north]south:$\Delta x$}] at (A.east) {$a\Delta x$};
\node (D) [anchor=south, rectangle, draw, fill=color2!50, minimum width=1cm, minimum height=1cm] at (C.north) {$\Delta x^2$};
\end{tikzpicture}
}}
\end{center}

\end{frame}


\subsection{Algebra of derivatives}
%---------------------------------------------------------------------slide----
\begin{frame}
\frametitle{Properties of the derivative}
If $y=c$, is a constant function, then $y'=0$ at any point.

If $y=x$, is the identity function, then  $y'=1$ at any point.

If $u=f(x)$ and $v=g(x)$ are two differentiable functions, then 
\begin{itemize}
\item $(u+v)'=u'+v'$
\item $(u-v)'=u'-v'$
\item $(u\cdot v)'=u'\cdot v+ u\cdot v'$
\item $\left(\dfrac{u}{v}\right)'=\dfrac{u'\cdot v-u\cdot v'}{v^2}$
\end{itemize}
\end{frame}


%---------------------------------------------------------------------slide----
\begin{frame}
\frametitle{Derivative of a composite function}
\framesubtitle{The chain rule}
\begin{theorem}[Chain rule] If the function $y=f\circ g$ is the composition of two functions $y=f(z)$ and $z=g(x)$, then
\[ 
(f\circ g)'(x)=f'(g(x))g'(x).
\]
\end{theorem}

It's easy to proof this using the Leibniz notation
\[
\frac{dy}{dx}=\frac{dy}{dz}\frac{dz}{dx}=f'(z)g'(x)=f'(g(x))g'(x).
\]

\structure{\textbf{Example}} If $f(z)=\sin z$ and $g(x)=x^2$, then $f\circ g(x)=\sin(x^2)$. Applying the chain rule the derivative of the composite function is
\[
(f\circ g)'(x)=f'(g(x))g'(x) = \cos(g(x)) 2x = \cos(x^2)2x.
\]
On the other hand, $g\circ f(z)= (\sin z)^2$, and applying the chain rule again, its derivative is
\[
(g\circ f)'(z)=g'(f(z))f'(z) = 2f(z)\cos z = 2\sin z\cos z.
\]
\end{frame}


%---------------------------------------------------------------------slide----
\begin{frame}
\frametitle{Derivative of the inverse of a function}
\begin{theorem}[Derivative of the inverse function]
Given a function $y=f(x)$ with inverse $x=f^{-1}(y)$, then 
\[
\left(f^{-1}\right)'(y)=\frac{1}{f'(x)}=\frac{1}{f'(f^{-1}(y))},
\]
provided that $f$ is differentiable at $f^{-1}(y)$ and $f'(f^{-1}(y))\neq 0$.
\end{theorem}

It's easy to proof this using the Leibniz notation
\[
\frac{dx}{dy}=\frac{1}{dy/dx}=\frac{1}{f'(x)}=\frac{1}{f'(f^{-1}(y))}
\]
\end{frame}


%---------------------------------------------------------------------slide----
\begin{frame}
\frametitle{Derivative of the inverse of a function}
\framesubtitle{Example}
The inverse of the exponential function $y=f(x)=e^x$ is the natural logarithm $x=f^{-1}(y)=\ln y$, so that we can compute the derivative of the natural logarithm using the previous theorem and we get
\[
\left(f^{-1}\right)'(y)=\frac{1}{f'(x)}=\frac{1}{e^x}=\frac{1}{e^{\ln y}}=\frac{1}{y}.
\]

\structure{\textbf{Example}} Sometimes is easier to apply the chain rule to compute the derivative of the inverse of a function. 
In this example, as $\ln x$ is the inverse of $e^x$, we know that $e^{\ln x}=x$, so that differentiating both sides and applying the chain rule to the left side we get
\[
(e^{\ln x})'=x' \Leftrightarrow e^{\ln x}(\ln(x))' = 1 \Leftrightarrow (\ln(x))'=\frac{1}{e^{\ln x}}=\frac{1}{x}.
\]
\end{frame}



\subsection{Kinematics}
% ---------------------------------------------------------------------slide----
\begin{frame}
\frametitle{Linear motion}
Assume that the function $y=f(t)$ describes the position of an object moving in the real line at every time $t$.
Taking as reference the coordinates origin $O$ and the unitary vector $\mathbf{i}=(1)$, we can represent the position of the moving object $P$ at every moment $t$ with a vector $\vec{OP}=x\mathbf{i}$ where $x=f(t)$.
\begin{center}
\tikzsetnextfilename{derivatives_1_variable/linear_motion}
% Author: Alfredo Sánchez Alberca (asalber@ceu.es)
\begin{tikzpicture}
\draw (0,0) -- (2,0) node[midway, above=0.5] {Time};
\draw (4,0) -- (8,0) node[midway, above=0.5] {Position} node[right] {$\mathbb{R}$};
\draw (1,0.1) -- (1,-0.1) node[anchor=north] (A) {$t$};
\draw (5,0.1) -- (5,-0.1) node[anchor=north] {$O$};
\draw (6,0.1) -- (6,-0.1) node[anchor=north] {$1$};
\draw [->, color1] (5,0) -- (7,0);
\draw [->, color2] (5,0) -- (6,0) node[midway, above] {\color{color2}$\mathbf{i}$};
\fill (7,0) circle (1.2pt) node[above] {$P$} node[anchor=north, xshift=2.5ex] (P) {$x=f(t)$};
\path [->, dashed] (A) edge[bend right] node[above] {$f$} (P);
\end{tikzpicture}

\end{center}

\structure{\textbf{Observation}} It also makes sense when $f$ measures other magnitudes as the temperature of a body, the concentration of a gas, or the quantity of substance in a chemical reaction at every moment $t$.
\end{frame}


% ---------------------------------------------------------------------slide----
\begin{frame}
\frametitle{Kinematic interpretation of the average rate of change}
In this context, if we take the instants $t=t_0$ and $t=t_0+\Delta t$, both in $\mbox{Dom}(f)$, the vector
\[
\mathbf{v}_m=\frac{f(t_0+\Delta t)-f(t_0)}{\Delta t}
\]
is known as the \emph{average velocity} of the trajectory $f$ in the interval $[t_0, t_0+\Delta t]$.

\structure{\textbf{Example}}
A vehicle makes a trip from Madrid to Barcelona.
Let $f(t)$ be the function that determine the position of the vehicle at every moment $t$.
If the vehicle departs from Madrid (km 0) at 8:00 and arrive to Barcelona (km 600) at 14:00, then the average velocity
of the vehicle in the path is
\[
\mathbf{v}_m=\frac{f(14)-f(8)}{14-8}=\frac{600-0}{6} = 100 km/h.
\]
\end{frame}


% ---------------------------------------------------------------------slide----
\begin{frame}
\frametitle{Kinematic interpretation of the derivative}
In the same context of the linear motion, the derivative of the function $f(t)$ at the moment $t_0$ is the vector
\[
\mathbf{v}=f'(t_0)=\lim_{\Delta t\rightarrow 0}\frac{f(t_0+\Delta t)-f(t_0)}{\Delta t},
\]
that is known, as long as the limit exists, as the \emph{instantaneous velocity} or simply \emph{velocity} of the trajectory $f$ at moment $t_0$.

That is, the derivative of the object position with respect to time is a vector field that is called \emph{velocity along the trajectory $f$}.

\structure{\textbf{Example}}
Following with the previous example, what indicates the speedometer at any instant is the modulus of the instantaneous velocity vector at that moment.
\end{frame}


% ---------------------------------------------------------------------slide----
\begin{frame}
\frametitle{Generalization to curvilinear motion}
The notion of derivative as a velocity along a trajectory in the real line can be generalized to a trajectory in any euclidean space $\mathbb{R}^n$.

In case of a two dimensional space $\mathbb{R}^2$, if $f(t)$ describes the position of a moving object in the real plane at any time $t$, taking as reference the coordinates origin $O$ and the unitary vectors $\{\mathbf{i}=(1,0),\mathbf{j}=(0,1)\}$, we can represent the position of the moving object $P$ at every moment $t$ with a vector $\vec{OP}=x(t)\mathbf{i}+y(t)\mathbf{j}$, where the coordinates 
\[
\begin{cases}
x=x(t)\\
y=y(t)
\end{cases}
\quad
t\in \mbox{Dom}(f)
\]
are known as \emph{coordinate functions} of $f$ and denoted $f(t)=(x(t),y(t))$.

\begin{center}
\tikzsetnextfilename{derivatives_1_variable/curvilinear_motion}
\mode<article>{% Author: Alfredo Sánchez Alberca (asalber@ceu.es)
\begin{tikzpicture}
\draw (-5,1) -- (-3,1) node[midway, above=0.5] {Time};
\draw (-4,0.9) -- (-4,1.1) node[anchor=south] (A) {$t$};
\begin{axis}[
    gen2dfun, 
    xmin=-0.1, xmax=3,
    ymin=-0.1, ymax=3,
    axis equal=true,  
    xtick={1,2},
    xticklabels={$1$,$x(t)$},
    ytick={1,2.041471},
    yticklabels={$1$,$y(t)$}, 
    height=2cm,
    clip=false,
    ]
    \addplot+[domain=-0.2:3, smooth, name path=F] {sin(deg(x-1))+1.2};
    \coordinate (O) at (0,0);
    \coordinate (I) at (1,0);
    \coordinate (J) at (0,1);
    \draw [->, color1] (O) -- (I) node[above] {$\mathbf{i}$};
    \draw [->, color1] (O) -- (J) node[right] {$\mathbf{j}$};
    \coordinate (P) at (2,2.041471);
    \fill (P) circle (1.2pt) node[anchor=south west] (Plabel) {$P=f(t)$};
    \draw[gray, dotted] (P) -- (P|-O);
    \draw[gray, dotted] (P) -- (P-|O);
    \draw[->, color2] (O) -- (P);
\end{axis};
\node at (1.2,2.5) {Position};
\path [->, dashed] (A) edge[bend left=20] node[above] {$f$} (Plabel);
\end{tikzpicture}
}
\mode<presentation>{\scalebox{0.8}{% Author: Alfredo Sánchez Alberca (asalber@ceu.es)
\begin{tikzpicture}
\draw (-5,1) -- (-3,1) node[midway, above=0.5] {Time};
\draw (-4,0.9) -- (-4,1.1) node[anchor=south] (A) {$t$};
\begin{axis}[
    gen2dfun, 
    xmin=-0.1, xmax=3,
    ymin=-0.1, ymax=3,
    axis equal=true,  
    xtick={1,2},
    xticklabels={$1$,$x(t)$},
    ytick={1,2.041471},
    yticklabels={$1$,$y(t)$}, 
    height=2cm,
    clip=false,
    ]
    \addplot+[domain=-0.2:3, smooth, name path=F] {sin(deg(x-1))+1.2};
    \coordinate (O) at (0,0);
    \coordinate (I) at (1,0);
    \coordinate (J) at (0,1);
    \draw [->, color1] (O) -- (I) node[above] {$\mathbf{i}$};
    \draw [->, color1] (O) -- (J) node[right] {$\mathbf{j}$};
    \coordinate (P) at (2,2.041471);
    \fill (P) circle (1.2pt) node[anchor=south west] (Plabel) {$P=f(t)$};
    \draw[gray, dotted] (P) -- (P|-O);
    \draw[gray, dotted] (P) -- (P-|O);
    \draw[->, color2] (O) -- (P);
\end{axis};
\node at (1.2,2.5) {Position};
\path [->, dashed] (A) edge[bend left=20] node[above] {$f$} (Plabel);
\end{tikzpicture}
}}
\end{center}
\end{frame}


% ---------------------------------------------------------------------slide----
\begin{frame}
\frametitle{Velocity of a curvilinear motion in the plane}
In the context of a trajectory $f(t)=(x(t),y(t))$ in the real plane $\mathbb{R}^2$, the derivative of the function $f(t)$ at the moment $t_0$ is the vector
\[
\mathbf{v} = \lim_{\Delta t\rightarrow 0} \frac{f(t_0+\Delta t)-f(t_0)}{\Delta t},
\]
that is known, as long as the limit exists, as the \emph{velocity} of the trajectory $f$ at moment $t_0$.

As $f(t)=(x(t),y(t))$,
\begin{align*}
f'(t_0)&=\lim_{\Delta t\rightarrow 0} \frac{f(t_0+\Delta t)-f(t_0)}{\Delta t} = \lim_{\Delta t\rightarrow 0} \frac{(x(t_0+\Delta t),y(t_0+\Delta t))-(x(t_0),y(t_0))}{\Delta t} =\\
&=  \lim_{\Delta t\rightarrow 0} \left(\frac{x(t_0+\Delta t)-x(t_0)}{\Delta t},\frac{y(t_0+\Delta t)-y(t_0)}{\Delta t}\right) =\\
&= \left(\lim_{\Delta t\rightarrow 0}\frac{x(t_0+\Delta t)-x(t_0)}{\Delta t},\lim_{\Delta t\rightarrow 0}\frac{y(t_0+\Delta t)-y(t_0)}{\Delta t}\right) = (x'(t_0),y'(t_0)).
\end{align*}
Thus, 
\[
\mathbf{v} = x'(t_0)\mathbf{i}+y'(t_0)\mathbf{j}.
\]
\end{frame}


% ---------------------------------------------------------------------slide----
\begin{frame}
\frametitle{Velocity of a curvilinear motion in the plane}
\framesubtitle{Example}
Given the trajectory $f(t) = (\cos t,\sin t)$, $t\in \mathbb{R}$, whose image is the unit circumference centered in the coordinate origin, its coordinate functions are $x(t) = \cos t$, $y(t) = \sin t$, $t\in \mathbb{R}$, and its velocity is
\[
\mathbf{v}=f'(t)=(x'(t),y'(t))=(-\sin t, \cos t).
\]
In the moment $t=\pi/4$, the object is in position $f(\pi/4) = (\cos(\pi/4),\sin(\pi/4)) =(\sqrt{2}/2,\sqrt{2}/2)$
and it is moving with a velocity $\mathbf{v}=f'(\pi/4)=(-\sin(\pi/4),\cos(\pi/4))=(-\sqrt{2}/2,\sqrt{2}/2)$.
\begin{center}
\tikzsetnextfilename{derivatives_1_variable/circumference_trajectory}
\mode<article>{% Author: Alfredo Sánchez Alberca (asalber@ceu.es)
\begin{tikzpicture}
\begin{axis}[
    2dfun, 
    xmin=-1.5, xmax=1.5,
    ymin=-1.5, ymax=1.5,
    axis equal=true,  
    height=3cm,
    clip=false,
    ]
	\addplot+[domain=0:2*pi, samples=200, smooth]({cos(deg(x))},{sin(deg(x))});
	\coordinate (P) at (0.7071068,0.7071068); 
	\fill (P) circle (1.2pt) node [anchor=south west] {$P=f(\pi/4)$};
	\draw [->, color2] (P) -- (0,1.4142);
\end{axis};
\end{tikzpicture}
}
\mode<presentation>{\scalebox{0.8}{% Author: Alfredo Sánchez Alberca (asalber@ceu.es)
\begin{tikzpicture}
\begin{axis}[
    2dfun, 
    xmin=-1.5, xmax=1.5,
    ymin=-1.5, ymax=1.5,
    axis equal=true,  
    height=3cm,
    clip=false,
    ]
	\addplot+[domain=0:2*pi, samples=200, smooth]({cos(deg(x))},{sin(deg(x))});
	\coordinate (P) at (0.7071068,0.7071068); 
	\fill (P) circle (1.2pt) node [anchor=south west] {$P=f(\pi/4)$};
	\draw [->, color2] (P) -- (0,1.4142);
\end{axis};
\end{tikzpicture}
}}
\end{center}
Observe that the module of the velocity vector is always 1 as
$|\mathbf{v}|=\sqrt{(-\sin t)^2+(\cos t)^2}=1$.
\end{frame}



\subsection{Tangent line to a trajectory}
% ---------------------------------------------------------------------slide----
\begin{frame}
\frametitle{Tangent line to a trajectory in the plane}
\framesubtitle{Vectorial equation}
Given a trajectory $f(t)$ in the real plane, the vectors that are parallel to the velocity $\mathbf{v}$ at a moment $t_0$ are called \emph{tangent vectors} to the trajectory $f$ at the moment $t_0$, and the line passing through $P=f(t_0)$ directed by $\mathbf{v}$ is the tangent line to $f$ at the moment $t_0$.

\begin{definition}[Tangent line to a trajectory]
Given a trajectory $f(t)$ in the real plane $\mathbb{R}^2$, the \emph{tangent line} to $f$ at $t_0$ is the line with equation
\begin{align*}
l:(x,y) &= f(t_0)+tf'(t_0) = (x(t_0),y(t_0))+t(x'(t_0),y'(t_0))\\
& = (x(t_0)+tx'(t_0),y(t_0)+ty'(t_0)).
\end{align*}
\end{definition}
\end{frame}


% ---------------------------------------------------------------------slide----
\begin{frame}
\frametitle{Tangent line to a trajectory in the plane}
\framesubtitle{Example}
We have seen that for the trajectory $f(t) = (\cos t,\sin t)$, $t\in \mathbb{R}$, whose image is unit circumference at the coordinate origin, the object position at the moment $t=\pi/4$ is $f(\pi/4)=(\sqrt{2}/2,\sqrt{2}/2)$ and its velocity $\mathbf{v}=(-\sqrt{2}/2,\sqrt{2}/2)$. 
Thus the equation of the tangent line to $f$ at that moment is 
\[
l: (x,y) = f(\pi/4)+t\mathbf{v} =
\left(\frac{\sqrt{2}}{2},\frac{\sqrt{2}}{2}\right)+t\left(\frac{-\sqrt{2}}{2},\frac{\sqrt{2}}{2}\right) =
\left(\frac{\sqrt{2}}{2}-t\frac{\sqrt{2}}{2},\frac{\sqrt{2}}{2}+t\frac{\sqrt{2}}{2}\right).
\]
\end{frame}


% ---------------------------------------------------------------------slide----
\begin{frame}
\frametitle{Tangent line to a trajectory in the plane}
\framesubtitle{Cartesian and point-slope equations}
From the vectorial equation of the tangent to a trajectory $f(t)$ at the moment $t=t_0$ we can get the coordinate functions
\[
\begin{cases}
x=x(t_0)+tx'(t_0)\\
y=y(t_0)+ty'(t_0)
\end{cases}
\quad t\in \mathbb{R},
\]
and solving for $t$ and equalling both equations we get the \emph{Cartesian equation} of the tangent
\[
\frac{x-x(t_0)}{x'(t_0)}=\frac{y-y(t_0)}{y'(t_0)},
\]
if $x'(t_0)\neq 0$ and $y'(t_0)\neq 0$.

From this equation is easy to get the \emph{point-slope equation} of the tangent
\[
y-y(t_0)=\frac{y'(t_0)}{x'(t_0)}(x-x(t_0)).
\]
\end{frame}


% ---------------------------------------------------------------------slide----
\begin{frame}
\frametitle{Tangent line to a trajectory in the plane}
\framesubtitle{Example of Cartesian and point-slope equations}
Using the vectorial equation of the tangent of the previous example
\[
l: (x,y)=\left(\frac{\sqrt{2}}{2}-t\frac{\sqrt{2}}{2},\frac{\sqrt{2}}{2}+t\frac{\sqrt{2}}{2}\right),
\]
its Cartesian equation is 
\[
\frac{x-\sqrt{2}/2}{-\sqrt{2}/2} = \frac{y-\sqrt{2}/2}{\sqrt{2}/2}
\]
and the point-slope equation is 
\[
y-\sqrt{2}/2 = \frac{-\sqrt{2}/2}{\sqrt{2}/2}(x-\sqrt{2}/2) \Rightarrow y=-x+\sqrt{2}.
\]
\end{frame}


% ---------------------------------------------------------------------slide----
\begin{frame}
\frametitle{Normal line to a trajectory in the plane}
We have seen that the tangent line to a trajectory $f(t)$ at $t_0$ is the line passing through the point 
$P=f(t_0)$ directed by the velocity vector $\mathbf{v}=f'(t_0)=(x'(t_0),y'(t_0))$. 
If we take as direction vector a vector orthogonal to $\mathbf{v}$, we get another line that is known as \emph{normal line} to $f$ at moment $t_0$.
\begin{definition}[Normal line to a trajectory]
Given a trajectory $f(t)$ in the real plane $\mathbb{R}^2$, the \emph{normal line} to $f$ at moment $t=t_0$ is the line with equation
\[
l: (x,y)=(x(t_0),y(t_0))+t(y'(t_0),-x'(t_0)) = (x(t_0)+ty'(t_0),y(t_0)-tx'(t_0)).
\]
\end{definition}
The Cartesian equation is 
\[
\frac{x-x(t_0)}{y'(t_0)} = \frac{y-y(t_0)}{-x'(t_0)},
\]
and the point-slope equation is
\[
y-y(t_0) = \frac{-x'(t_0)}{y'(t_0)}(x-x(t_0)).
\]
The normal line is always perpendicular to the tangent line as their direction vectors are orthogonal. 
\end{frame}


% ---------------------------------------------------------------------slide----
\begin{frame}
\frametitle{Normal line to a trajectory in the plane}
\framesubtitle{Example}
Considering again the trajectory of the unit circumference $f(t) = (\cos t,\sin t)$, $t\in \mathbb{R}$, the normal line to $f$ at moment $t=\pi/4$ is
\begin{align*}
l: (x,y)&=(\cos(\pi/2),\sin(\pi/2))+t(\cos(\pi/2),\sin(\pi/2)) =\\
&= \left(\frac{\sqrt{2}}{2},\frac{\sqrt{2}}{2}\right)+t\left(\frac{\sqrt{2}}{2},\frac{\sqrt{2}}{2}\right)
=\left(\frac{\sqrt{2}}{2}+t\frac{\sqrt{2}}{2},\frac{\sqrt{2}}{2}+t\frac{\sqrt{2}}{2}\right),
\end{align*}
the Cartesian equation is 
\[
\frac{x-\sqrt{2}/2}{\sqrt{2}/2} = \frac{y-\sqrt{2}/2}{\sqrt{2}/2},
\]
and the point-slope equation is 
\[
y-\sqrt{2}/2 = \frac{\sqrt{2}/2}{\sqrt{2}/2}(x-\sqrt{2}/2) \Rightarrow y=x.
\]
% \begin{center}
% \scalebox{0.8}{\input{img/calculo_diferencial_1_variable/circunferencia_tangente_normal}}
% \end{center}
\end{frame}


% ---------------------------------------------------------------------slide----
\begin{frame}
\frametitle{Tangent and normal lines to a function}
A particular case of tangent and normal lines to a trajectory are the tangent and normal lines to a function of one real variable. 
For every function $y=f(x)$, the trajectory that trace its graph is
\[
g(x) = (x,f(x))  \quad x\in \mathbb{R},
\]
and its velocity is 
\[
g'(x) = (1,f'(x)),
\]
so that the tangent line to $g$ at the moment $x_0$ is
\[
\frac{x-x_0}{1} = \frac{y-f(x_0)}{f'(x_0)} \Rightarrow y-f(x_0) = f'(x_0)(x-x_0),
\]
and the normal line is 
\[
\frac{x-x_0}{f'(x_0)} = \frac{y-f(x_0)}{-1} \Rightarrow y-f(x_0) = \frac{-1}{f'(x_0)}(x-x_0),
\]
\end{frame}


% ---------------------------------------------------------------------slide----
\begin{frame}
\frametitle{Tangent and normal lines to a function}
\framesubtitle{Example}
Given the function $y=x^2$, the trajectory that traces the its graph is $g(x)=(x,x^2)$ and its velocity is 
$g'(x)=(1,2x)$. At the moment $x=1$ the trajectory passes through the point $(1,1)$ with a velocity $(1,2)$.
Thus, the tangent line at that moment is 
\[
\frac{x-1}{1} = \frac{y-1}{2} \Rightarrow y-1 = 2(x-1) \Rightarrow y = 2x-1,
\]
and the normal line is 
\[
\frac{x-1}{2} = \frac{y-1}{-1} \Rightarrow y-1 = \frac{-1}{2}(x-1) \Rightarrow y = \frac{-x}{2}+\frac{3}{2}.
\]
\begin{center}
\tikzsetnextfilename{derivatives_1_variable/parable_tangent_normal}
% Author: Alfredo Sánchez Alberca (asalber@ceu.es)
\begin{tikzpicture}
  \begin{axis}[
    2dfun, 
    xmin=-2, xmax=2,
    ymin=-0.2, ymax=3,
    axis equal=true,
    clip=false,  
    height=2cm,
    ]
    \addplot+[domain=-2:2] {x^2} node[anchor=south] {$f(x)=x^2$};
    \addplot+[domain=0.2:2] {2*x-1} node[anchor=west] {Tangent $y=2x-1$};
    \addplot+[domain=-2:2] {-x/2+3/2} node[anchor=west] {Normal $y=-\frac{x}{2}+\frac{3}{2}$};
    \coordinate (A) at (1,1);
    \fill (A) circle (1.2pt);
  \end{axis}
\end{tikzpicture}

\end{center}
\end{frame}


% ---------------------------------------------------------------------slide----
\begin{frame}
\frametitle{Tangent line to a trajectory in the space}
The concept of tangent line to a trajectory in can be easily extended from the real plane to the three-dimensional space $\mathbb{R}^3$.

If $f(t)=(x(t),y(t),z(t))$, $t\in \mathbb{R}$, is a trajectory in the real space $\mathbb{R}^3$, then at the moment $t_0$, the moving object that follows this trajectory will be at the position $P=(x(t_0),y(t_0),z(t_0))$ with a velocity $\mathbf{v}=f'(t)=(x'(t),y'(t),z'(t))$.
Thus, the tangent line to $f$ at this moment have the following vectorial equation
\begin{align*}
l&: (x,y,z)=(x(t_0),y(t_0),z(t_0))+t(x'(t_0),y'(t_0),z'(t_0)) =\\
&= (x(t_0)+tx'(t_0),y(t_0)+ty'(t_0),z(t_0)+tz'(t_0)),
\end{align*}
and the Cartesian equations are 
\[
\frac{x-x(t_0)}{x'(t_0)}=\frac{y-y(t_0)}{y'(t_0)}=\frac{z-z(t_0)}{z'(t_0)},
\]
provided that $x'(t_0)\neq 0$, $y'(t_0)\neq 0$ y $z'(t_0)\neq 0$.
\end{frame}


% ---------------------------------------------------------------------slide----
\begin{frame}
\frametitle{Tangent line to a trajectory in the space}
\framesubtitle{Example}
Given the trajectory $f(t)=(\cos t, \sin t, t)$, $t\in \mathbb{R}$ in the real space, at the moment $t=\pi/2$ the trajectory passes through the point
\[
f(\pi/2)=(\cos(\pi/2),\sin(\pi/2),\pi/2)=(0,1,\pi/2),
\]
with a velocity
\[
\mathbf{v}=f'(\pi/2)=(-\sin(\pi/2),\cos(\pi/2), 1)=(-1,0,1),
\] 
and the tangent line to $f$ at that moment is 
\[
l:(x,y,z)=(0,1,\pi/2)+t(-1,0,1) = (-t,1,t+\pi/2).
\]
\begin{center}
\tikzsetnextfilename{derivatives_1_variable/tangent_trajectory_space}
% Author: Alfredo Sánchez Alberca (asalber@ceu.es)
\begin{tikzpicture}
  \begin{axis}[
  3dfun,
  clip=false,
  height=3cm,
  ]
    \addplot3+[domain=0:pi, samples y=0] ({cos(deg(x))}, {sin(deg(x))}, {x});
    \coordinate (P) at (0,1,pi/2);
    \fill (P) circle (1.2pt) node[anchor=west] {$P=f(\pi/2)=(0,1,\pi/2)$};
    \addplot3+[domain=-pi/2:pi/2, samples y=0] ({-x}, {1}, {x+pi/2});
    \draw [->, color3] (P) -- (-1,1,2.57) node[anchor=west] {$\textbf{v}$};
  \end{axis}
\end{tikzpicture}

\end{center}
\end{frame}


\subsection{Analysis of functions}
%---------------------------------------------------------------------slide----
\begin{frame}
\frametitle{Analysis of functions: increase and decrease}
The main application of derivatives is to determine the variation (increase or decrease) of functions. 
For that we use the sign of the first derivative.  
\begin{theorem}
Let $f(x)$ be a function with first derivative in an interval $I\subseteq \mathbb{R}$.
\begin{itemize}
\item If $\forall x\in I\ f'(x)\geq 0$ then $f$ is increasing on $I$.
\item If $\forall x\in I\ f'(x)\leq 0$ then $f$ is decreasing on $I$.
\end{itemize}
\end{theorem}
If $f'(x_0)=0$ then $x_0$ is known as a \emph{stationary point} and the function in non-increasing and non-decreasing at that point. 
\structure{\textbf{Example}}
The function $f(x)=x^3$ is increasing on $\mathbb{R}$ as $\forall x\in \mathbb{R}\
f'(x)\geq 0$. \vskip .5cm
\structure{\textbf{Observation}} {A function can be increasing or decreasing on an interval and not have first derivative.}
\end{frame}


%---------------------------------------------------------------------slide----
\begin{frame}
\frametitle{Analysis of functions: increase and decrease}
\framesubtitle{Example}
Let's analyze the increase and decrease of the function $f(x)=x^4-2x^2+1$. 
Its first derivative is $f'(x)=4x^3-4x$.
\begin{center}
\tikzsetnextfilename{derivatives_1_variable/increase_analysis}
\scalebox{0.9}{% Author: Alfredo Sánchez Alberca (asalber@ceu.es)
\begin{tikzpicture}
  \begin{axis}[
    2dfun, 
    xmin=-2, xmax=2,
    ymin=-2, ymax=2,
    axis equal=true,
    clip=false, 
    width=5cm,
    ]
    \addplot+[domain=-1.5:1.5, smooth, name path=F] {x^4-2*x^2+1} node[anchor=south west] {$f(x)=x^4-2x^2+1$};
    \addplot+[domain=-1.2:1.2, smooth, name path=D] {4*x^3-4*x)} node[anchor=south west] {$f'(x)=4x^3-4x$};
    \node[anchor=east, visible on=<2->] at (-2,-2.7) {Increase $f(x)$};
    \node[anchor=east, visible on=<2->] at (-2,-3.5) {Sign $f'(x)$};
%     \draw[->] (-2.2,-2.8) -- (2.2,-2.8) node[anchor = west] {$x$};
%     \foreach \x in {-2,...,2}
%    		\draw (\x cm, 1pt) -- (\x cm, -1pt) node[anchor = north] {$\x$};
	\only
    \draw<3->[dashed, gray] (-1,2) -- (-1,-3.5);
    \draw<3->[dashed, gray] (0,2) -- (0,-3.5);
    \draw<3->[dashed, gray] (1,2) -- (1,-3.5);
    \node<3->[color2] at (-1,-3.5) {$0$};
    \node<3->[color2] at (0,-3.5) {$0$};
    \node<3->[color2] at (1,-3.5) {$0$};
    \node<4->[color2] at (-1.5,-3.5) {$-$};
    \draw<4-> [->, color1] (-1.5, -2.5) -- (-1.5,-2.8);
    \node<5->[color2] at (-0.5,-3.5) {$+$};
    \draw<5-> [->, color1] (-0.5, -2.8) -- (-0.5,-2.5);
    \node<6->[color2] at (0.5,-3.5) {$-$};
    \draw<6-> [->, color1] (0.5, -2.5) -- (0.5,-2.8);   
    \node<7->[color2] at (1.5,-3.5) {$+$};
    \draw<7-> [->, color1] (1.5, -2.8) -- (1.5,-2.5);
  \end{axis}
  \begin{axis}[visible on=<2->,
    2dfun, 
    xmin=-2.31, xmax=2.31,
    ymin=-0.1, ymax=0.1,
    hide y axis,  
    axis equal=true,
  	at={(0,-30)},
  	width=5cm, 
  ]
  \end{axis}
\end{tikzpicture}
}
\end{center}
\end{frame}


%---------------------------------------------------------------------slide----
\begin{frame}
\frametitle{Analysis of functions: relative extrema}
As a consequence of the previous result we can also use the first derivative to determine the relative extrema of a function. 
\begin{theorem}[First derivative test]
Let $f(x)$ be a function with first derivative in an interval $I\subseteq \mathbb{R}$ and let $x_0\in I$ be a stationary point of $f$ ($f'(x_0)=0$).
\begin{itemize}
\item  If $f'(x)>0$ on an open interval extending left from $x_0$ and $f'(x)<0$ on an open interval extending right from $x_0$, then $f$ has a \emph{relative maximum} at $x_0$.
\item  If $f'(x)<0$ on an open interval extending left from $x_0$ and $f'(x)>0$ on an open interval extending right from $x_0$, then $f$ has a \emph{relative minimum} at $x_0$.
\item If $f'(x)$ has the same sign on both an open interval extending left from $x_0$ and an open interval extending right from $x_0$, then $f$ has an \emph{inflection point} at $x_0$.
\end{itemize}
\end{theorem}

\structure{\textbf{Observation}} \emph{A vanishing derivative is a necessary but not sufficient condition for the function to have a relative extrema at a point.}

\structure{\textbf{Example}} The function $f(x)=x^3$ have derivative $f'(x)=3x^2$ and have a stationary point at $x=0$.
However it doesn't have a relative extrema at that point, but an inflection point.
\end{frame}


%---------------------------------------------------------------------slide----
\begin{frame}
\frametitle{Analysis of functions: relative extrema}
\framesubtitle{Example}
Consider again the function $f(x)=x^4-2x^2+1$ and let's analyze its relative extrema now. 
Its first derivative is $f'(x)=4x^3-4x$.
\begin{center}
\tikzsetnextfilename{derivatives_1_variable/extrema_analysis}
\scalebox{0.9}{% Author: Alfredo Sánchez Alberca (asalber@ceu.es)
\begin{tikzpicture}[trim axis left, trim axis right]
  \begin{axis}[
    2dfun, 
    xmin=-2, xmax=2,
    ymin=-2, ymax=2,
    axis equal=true,
    clip=false, 
    width=5cm,
    ]
    \addplot+[domain=-1.5:1.5, smooth, name path=F] {x^4-2*x^2+1} node[anchor=south west] {$f(x)=x^4-2x^2+1$};
    \addplot+[domain=-1.2:1.2, smooth, name path=D] {4*x^3-4*x)} node[anchor=south west] {$f'(x)=4x^3-4x$};
    \node[anchor=east] at (-2,-2.7) {Increase $f(x)$};
    \node[anchor=east] at (-2,-3.5) {Sign $f'(x)$};
    \node[anchor=east, visible on=<2->] at (-2,-4) {Extrema $f(x)$};
    \draw[dashed, gray] (-1,2) -- (-1,-3.5);
    \draw[dashed, gray] (0,2) -- (0,-3.5);
    \draw[dashed, gray] (1,2) -- (1,-3.5);
    \node[color2] at (-1,-3.5) {$0$};
    \node[color2] at (0,-3.5) {$0$};
    \node[color2] at (1,-3.5) {$0$};
    \node[color2] at (-1.5,-3.5) {$-$};
    \draw[->, color1] (-1.5, -2.5) -- (-1.5,-2.8);
    \node[color2] at (-0.5,-3.5) {$+$};
    \draw[->, color1] (-0.5, -2.8) -- (-0.5,-2.5);
    \node[color2] at (0.5,-3.5) {$-$};
    \draw[->, color1] (0.5, -2.5) -- (0.5,-2.8);   
    \node[color2] at (1.5,-3.5) {$+$};
    \draw[->, color1] (1.5, -2.8) -- (1.5,-2.5);
    \fill<3-> (-1,0) circle (1.2pt);
    \node<3->[color1] at (-1,-4) {Min};
    \fill<4-> (0,1) circle (1.2pt);
    \node<4->[color1] at (0,-4) {Max};
    \fill<5-> (1,0) circle (1.2pt);
    \node<5->[color1] at (1,-4) {Min};
  \end{axis};
  \begin{axis}[
    2dfun, 
    xmin=-2.31, xmax=2.31,
    ymin=-0.1, ymax=0.1,
    hide y axis,  
    axis equal=true,
  	at={(0,-30)},
  	width=5cm, 
  ]
  \end{axis}
\end{tikzpicture}
}
\end{center}
\end{frame}


%---------------------------------------------------------------------slide---
\begin{frame}
\frametitle{Analysis of functions: concavity}
The concavity of a function can be determined by de second derivative. 
\begin{theorem}
Let $f(x)$ be a function with second derivative in an interval $I\subseteq \mathbb{R}$.
\begin{itemize}
\item If $\forall x\in I\ f''(x)\geq 0$ then $f$ is concave up (convex) on $I$.
\item If $\forall x\in I\ f''(x)\leq 0$ then $f$ is concave down (concave) on $I$.
\end{itemize}
\end{theorem}

\structure{\textbf{Example}} The function $f(x)=x^2$ has second derivative $f''(x)=2>0$ $\forall x\in \mathbb{R}$, so it is concave up in all $\mathbb{R}$. 
\vskip .5cm
\structure{\textbf{Observation}} \emph{A function can be concave up or down and not have second derivative.}
\end{frame}


%---------------------------------------------------------------------slide----
\begin{frame}
\frametitle{Analysis of functions: concavity}
\framesubtitle{Example}
Let's analyze the concavity of the same function of previous examples $f(x)=x^4-2x^2+1$. 
Its second derivative is $f''(x)=12x^2-4$.
\begin{center}
\tikzsetnextfilename{derivatives_1_variable/concavity_analysis}
\scalebox{0.9}{% Author: Alfredo Sánchez Alberca (asalber@ceu.es)
\begin{tikzpicture}
  \begin{axis}[
    2dfun, 
    xmin=-2, xmax=2,
    ymin=-4, ymax=2,
    clip=false, 
    width=5cm,
  	]
    \addplot+[domain=-1.5:1.5, smooth, name path=F] {x^4-2*x^2+1} node[anchor=south west] {$f(x)=x^4-2x^2+1$};
    \addplot+[domain=-1.2:1.2, smooth, name path=D] {4*x^3-4*x)} node[anchor=south west] {$f'(x)=4x^3-4x$};
    \addplot+[domain=-0.7:0.7, smooth, name path=D] {12*x^2-4)} node[anchor=south west, pos=0.6] {$f'(x)=12x^2-4$};
    \node[anchor=east, visible on=<2->] at (-2,-4.6) {Concavity $f(x)$};
    \node[anchor=east, visible on=<2->] at (-2,-5.5) {Sign $f''(x)$};
    \draw<3->[dashed, gray] (-0.5773503,2) -- (-0.5773503,-5.3);
    \draw<3->[dashed, gray] (0.5773503,2) -- (0.5773503,-5.3);
    \node<3->[color3] at (-0.5773503,-5.5) {$0$};
    \node<3->[color3] at (0.5773503,-5.5) {$0$};
    \node<4->[color2] at (-1.2,-5.5) {$+$};
    \node<4->[color1] at (-1.2, -4.6) {$\cup$};
    \node<5->[color2] at (0,-5.5) {$-$};
    \node<5->[color1] at (0, -4.6) {$\cap$};
    \node<6->[color2] at (1.2,-5.5) {$+$};
    \node<6->[color1] at (1.2, -4.6) {$\cup$};
    \node[color1, visible on=<7->] at (-0.5773503,-6) {Inflection};
    \node[color1, visible on=<7->] at (0.5773503,-6) {Inflection};
  \end{axis};
  \begin{axis}[visible on=<2->,
  	2dfun, 
 	 xmin=-2, xmax=2,
  	ymin=-0.1, ymax=0.1,
  	hide y axis,  
  	at={(0,-135)},
  	width=5cm, 
  	]
  \end{axis}
\end{tikzpicture}
}
\end{center}
\end{frame}



\subsection{Taylor polynomials}
%---------------------------------------------------------------------slide----
\begin{frame}
\frametitle{Approximating a function by a polynomial}
Another useful application of the derivative is the approximation of functions by polynomials.

Polynomials are functions easy to calculate (sums and products) with very good properties:
\begin{itemize}
\item Defined in all the real numbers.
\item Continuous.
\item Differentiable of all orders with continuous derivatives.
\end{itemize}

\begin{block}{Goal}
Approximate a function $f(x)$ by a polynomial $p(x)$ near a value $x=x_0$.
\end{block}
\end{frame}


%---------------------------------------------------------------------slide----
\begin{frame}
\frametitle{Approximating a function by a polynomial of order 0}
A polynomial of grade 0 has equation
\[
p(x) = c_0,
\]
where $c_0$ is a constant.

As the polynomial should coincide with the function at $x_0$, it must satisfy
\[p(x_0) = c_0 = f(x_0).\]

Therefore, the polynomial of grade 0 that best approximate $f$ near $x_0$ is
\[p(x) = f(x_0).\]
\end{frame}


%---------------------------------------------------------------------slide----
\begin{frame}
\frametitle{Approximating a function by a polynomial of order 0}
\begin{center}
\tikzsetnextfilename{derivatives_1_variable/approximation_polynomial_0}
% Author: Alfredo Sánchez Alberca (asalber@ceu.es)
\begin{tikzpicture}
  \begin{axis}[
    gen2dfun, 
    xmin=0, xmax=4,
    ymin=0, ymax=4,
    xtick={2.5},
    xticklabel={$x_0$},
    ytick={1.5},
    yticklabels={$f(x_0)$}, 
    clip=false, 
    height=5cm,
  	]
    \addplot+[domain=0.5:3.6, smooth, name path=F] {2.7183^(x-2.5)+0.5} node[anchor=south west] {$f(x)$};
    \coordinate (A) at (2.5,1.5);
    \fill (A) circle (1.2pt);
    \draw[gray, dotted] (A) -- (A|-O);
    \draw[gray, dotted] (A) -- (A-|O);
    \addplot+[domain=0.5:3.6, smooth, name path=P, visible on=<2->] {1.5} node[anchor=west] {$p^0=f(x_0)$};
    \coordinate (B) at (3.5,3.2183);
    \draw<3->[gray, very thin] (3.5,0.06) -- (3.5,-0.06);
    \node<3-> at (3.5,-0.25) {$x_1$};
    \draw<3->[gray, very thin] (-0.06,3.2183) -- (0.06,3.2183);
    \node<3-> at (-0.25,3.2183) {$x_1$};
    \draw<3->[gray, dotted] (B) -- (B|-O);
    \draw<3->[gray, dotted] (B) -- (B-|O);
    \draw[|-|, visible on=<4->] (B) -- (B|-A) node[anchor=west, pos=0.3] {Approximation error} node[anchor=west, midway] {$e^0(x_1)=f(x_1)-p^0(x_1)$};
  \end{axis};
\end{tikzpicture}

\end{center}
\end{frame}


%---------------------------------------------------------------------slide----
\begin{frame}
\frametitle{Approximating a function by a polynomial of order 1}
A polynomial of grade 1 has equation
\[
p(x) = c_0+c_1x,
\]
but it can also be expressed 
\[
p(x) = c_0+c_1(x-x_0).
\]

Among all the polynomials of grade 1, the one that best approximate $f(x)$ near $x_0$ is that that meets the following conditions

\begin{enumerate}
\item $p$ and $f$ coincide at $x_0$: $p(x_0) = f(x_0)$,
\item $p$ and $f$ have the same rate of change at $x_0$: $p'(x_0) = f'(x_0)$.
\end{enumerate}

The last condition guarantee that $p$ and $f$ have approximately the same tendency, but it requires the function $f$ to be differentiable at $x_0$.
\end{frame}


%---------------------------------------------------------------------slide----
\begin{frame}
\frametitle{The tangent line: Best approximating polynomial of order 1}
Imposing the previous conditions we have
\begin{enumerate}
\item $p(x)=c_0+c_1(x-x_0) \Rightarrow p(x_0)=c_0+c_1(x_0-x_0)=c_0=f(x_0)$,
\item $p'(x)=c_1 \Rightarrow p'(x_0)=c_1=f'(x_0)$.
\end{enumerate}

Therefore, the polynomial of grade 1 that best approximates $f$ near $x_0$ is
\[
p(x) = f(x_0)+f '(x_0)(x-x_0),
\]
which turns out to be the tangent line to $f$ at $(x_0,f(x_0))$.
\end{frame}


%---------------------------------------------------------------------slide----
\begin{frame}
\frametitle{Approximating a function by a polynomial of order 1}
\begin{center}
\tikzsetnextfilename{derivatives_1_variable/approximation_polynomial_1}
% Author: Alfredo Sánchez Alberca (asalber@ceu.es)
\begin{tikzpicture}
  \begin{axis}[
    gen2dfun, 
    xmin=0, xmax=4,
    ymin=0, ymax=4,
    xtick={2.5},
    xticklabel={$x_0$},
    ytick={1.5},
    yticklabels={$f(x_0)$}, 
    clip=false, 
    height=5cm,
  	]
    \addplot+[domain=0.5:3.6, smooth, name path=F] {2.7183^(x-2.5)+0.5} node[anchor=south west] {$f(x)$};
    \coordinate (O);
    \coordinate (A) at (2.5,1.5);
    \fill (A) circle (1.2pt);
    \draw[gray, dotted] (A) -- (A|-O);
    \draw[gray, dotted] (A) -- (A-|O);
    \addplot+[domain=0.5:3.6, smooth, name path=P0, visible on=<2->] {1.5} node[anchor=west] {$p^0=f(x_0)$};
    \addplot+[domain=1:3.6, smooth, name path=P1, visible on=<3->] {x-1} node[anchor=west] {$p^1=f(x_0)+f'(x_0)(x-x_0)$};
    \coordinate (B) at (3.5,3.2183);
    \coordinate (C) at (3.5,2.5);
    \draw<4->[gray, very thin] (3.5,0.06) -- (3.5,-0.06);
    \node<4-> at (3.5,-0.25) {$x_1$};
    \draw<4->[gray, very thin] (-0.06,3.2183) -- (0.06,3.2183);
    \node<4->[anchor=east] at (-0.15,3.2183) {$f(x_1)$};
    \draw<4->[gray, very thin] (-0.06,2.5) -- (0.06,2.5);
    \node<4->[anchor=east] at (-0.15,2.5) {$p^1(x_1)$};
    \draw<4->[gray, dotted] (B) -- (B-|O);
    \draw<4->[gray, dotted] (B) -- (B|-O);
    \draw<4->[gray, dotted] (C) -- (C-|O);
    \draw[|-|, visible on=<5->] (B) -- (C) node[anchor=west, pos=0.1] {Approximation error} node[anchor=west, midway] {$e^1(x_1)=f(x_1)-p^1(x_1)$};
  \end{axis};
\end{tikzpicture}

\end{center}
\end{frame}


%---------------------------------------------------------------------slide----
\begin{frame}
\frametitle{Approximating a function by a polynomial of order 2}
A polynomial of grade 2 is a parable with equation
\[
p(x) = c_0+c_1x+c_2x^2,
\]
but it can also be expressed
\[
p(x) = c_0+c_1(x-x_0)+c_2(x-x_0)^2.
\]

Among all the polynomials of grade 2, the one that best approximate $f(x)$ near $x_0$ is that that meets the following conditions
\begin{enumerate}
\item $p$ and $f$ coincide at $x_0$: $p(x_0) = f(x_0)$,
\item $p$ and $f$ have the same rate of change at $x_0$: $p'(x_0) = f'(x_0)$.
\item $p$ and $f$ have the same concavity at $x_0$: $p''(x_0)=f''(x_0)$.
\end{enumerate}
The last condition requires the function $f$ to be differentiable twice at $x_0$.
\end{frame}


%---------------------------------------------------------------------slide----
\begin{frame}
\frametitle{Best approximating polynomial of order 2}
Imposing the previous conditions we have
\begin{enumerate}
\item $p(x)=c_0+c_1(x-x_0) \Rightarrow p(x_0)=c_0+c_1(x_0-x_0)=c_0=f(x_0)$,
\item $p'(x)=c_1 \Rightarrow p'(x_0)=c_1=f'(x_0)$.
\item $p''(x)=2c_2 \Rightarrow p''(x_0)=2c_2=f''(x_0) \Rightarrow c_2=\frac{f''(x_0)}{2}$.
\end{enumerate}

Therefore, the polynomial of grade 2 that best approximates $f$ near $x_0$ is
\[
p(x) = f(x_0)+f'(x_0)(x-x_0)+\frac{f''(x_0)}{2}(x-x_0)^2.
\]
\end{frame}


%---------------------------------------------------------------------slide----
\begin{frame}
\frametitle{Approximating a function by a polynomial of order 2}
\begin{center}
\tikzsetnextfilename{derivatives_1_variable/approximation_polynomial_2}
% Author: Alfredo Sánchez Alberca (asalber@ceu.es)
\begin{tikzpicture}[trim axis left, trim axis right]
  \begin{axis}[
    gen2dfun, 
    xmin=0, xmax=4,
    ymin=0, ymax=4,
    xtick={2.5},
    xticklabel={$a$},
    ytick={1.5},
    yticklabels={$f(a)$}, 
    clip=false, 
    height=5cm,
  	]
    \addplot+[domain=0.5:3.6, smooth, name path=F] {2.7183^(x-2.5)+0.5} node[anchor=south west] {$f(x)$};
    \coordinate (O);
    \coordinate (A) at (2.5,1.5);
    \fill (A) circle (1.2pt);
    \draw[gray, dotted] (A) -- (A|-O);
    \draw[gray, dotted] (A) -- (A-|O);
    \addplot+[domain=0.5:3.6, smooth, name path=P0, visible on=<2->] {1.5} node[anchor=west] {$p^0=f(a)$};
    \addplot+[domain=1:3.6, smooth, name path=P1, visible on=<3->] {x-1} node[anchor=west] {$p^1=f(a)+f'(a)(x-a)$};
    \addplot+[domain=0.5:3.6, smooth, name path=P1, visible on=<4->] {-1+x+(x-2.5)^2/2} node[anchor=west] {$p^2=f(a)+f'(a)(x-a)+\frac{f''(a)}{2}(x-a)^2$};
    \coordinate (B) at (3.5,3.2183);
    \coordinate (C) at (3.5,3);
    \draw<4->[gray, very thin] (3.5,0.05) -- (3.5,-0.05);
    \node<4-> at (3.5,-0.25) {$x$};
    \draw<4->[gray, very thin] (-0.06,3.2183) -- (0.06,3.2183);
    \node<4-> at (-0.25,3.2183) {$f(x)$};
    \draw<4->[gray, dotted] (B) -- (B|-O);
    \draw<4->[gray, dotted] (B) -- (B-|O);
    \draw[dashed, visible on=<5->] (B) -- (B|-A);
    \draw[decorate, decoration={brace, amplitude=4pt}, visible on=<5->] (B) -- (B|-A) node [pos=0.3, right, xshift=5pt] {Approximation error} node[midway, right, xshift=4pt] {$e^2(x)=f(x)-p^2(x)$};
  \end{axis};
\end{tikzpicture}

\end{center}
\end{frame}


%---------------------------------------------------------------------slide----
\begin{frame}
\frametitle{Approximating a function by a polynomial of order $n$}
A polynomial of grade $n$ has equation
\[
p(x) = c_0+c_1x+c_2x^2+\cdots +c_nx^n,
\]
but it can also be expressed
\[
p(x) = c_0+c_1(x-x_0)+c_2(x-x_0)^2+\cdots +c_n(x-x_0)^n.
\]

Among all the polynomials of grade 2, the one that best approximate $f(x)$ near $x_0$ is that that meets the following $n+1$ conditions
\begin{enumerate}
\item $p(x_0) = f(x_0)$,
\item $p'(x_0) = f'(x_0)$,
\item $p''(x_0)=f''(x_0)$,
\item[] $\cdots$
\item[n+1.] $p^{(n)}(x_0)=f^{(n)}(x_0)$.
\end{enumerate}

\alert{Observe that these conditions require the function $f$ to be differentiable $n$ times at $x_0$.}
\end{frame}


%---------------------------------------------------------------------slide----
\begin{frame}
\frametitle{Coefficients calculation for the best approximating polynomial of order $n$}
The successive derivatives of $p$ are 
\begin{align*}
p(x) &= c_0+c_1(x-x_0)+c_2(x-x_0)^2+\cdots +c_n(x-x_0)^n,\\
p'(x)& = c_1+2c_2(x-x_0)+\cdots +nc_n(x-x_0)^{n-1},\\
p''(x)& = 2c_2+\cdots +n(n-1)c_n(x-x_0)^{n-2},\\
\vdots\ \
\\
p^{(n)}(x)&= n(n-1)(n-2)\cdots 1 c_n=n!c_n.
\end{align*}

Imposing the previous conditions we have
\begin{enumerate}
\item $p(x_0) = c_0+c_1(x_0-x_0)+c_2(x_0-x_0)^2+\cdots +c_n(x_0-x_0)^n=c_0=f(x_0)$,
\item $p'(x_0) = c_1+2c_2(x_0-x_0)+\cdots +nc_n(x_0-x_0)^{n-1}=c_1=f'(x_0)$,
\item $p''(x_0) = 2c_2+\cdots +n(n-1)c_n(x_0-x_0)^{n-2}=2c_2=f''(x_0)\Rightarrow c_2=f''(x_0)/2$,
\item[] $\cdots$
\item[n+1.] $p^{(n)}(x_0)=n!c_n=f^{(n)}(x_0)=c_n=\frac{f^{(n)}(x_0)}{n!}$.
\end{enumerate}
\end{frame}


%---------------------------------------------------------------------slide----
\begin{frame}
\frametitle{Taylor polynomial of order $n$}
\begin{definition}[Taylor polynomial]
Given a function $f(x)$ differentiable $n$ times at $x_0$, the \emph{Taylor polynomial} of order $n$ of $f$ at $x_0$ is the polynomial with equation
\begin{align*}
p_{f,x_0}^n(x) &= f(x_0) + f'(x_0)(x-x_0) + \frac{f''(x_0)}{2}(x-x_0)^2 + \cdots + \frac{f^{(n)}(x_0)}{n!}(x-x_0)^n = \\ 
&= \sum_{i=0}^{n}\frac{f^{(i)}(x_0)}{i!}(x-x_0)^i.
\end{align*}
\end{definition}

The Taylor polynomial of order $n$ of $f$ at $x_0$ is the $n$th degree polynomial that best approximates $f$ near $x_0$, as is the only one that meets the previous conditions.
\end{frame} 


%---------------------------------------------------------------------slide----
\begin{frame}
\frametitle{Taylor polynomial calculation}
\framesubtitle{Example}
Let's approximate the function $f(x)=\log x$ near the value $1$ by a polynomial of order $3$.

The equation of the Taylor polynomial of order $3$ of $f$ at $x_0=1$ is
\[
p_{f,1}^3(x)=f(1)+f'(1)(x-1)+\frac{f''(1)}{2}(x-1)^2+\frac{f'''(1)}{3!}(x-1)^3.
\]
The derivatives of $f$ at $1$ up to order $3$ are
\[
\begin{array}{lll}
f(x)=\log x & \quad & f(1)=\log 1 =0,\\
f'(x)=1/x & & f'(1)=1/1=1,\\
f''(x)=-1/x^2 & & f''(1)=-1/1^2=-1,\\
f'''(x)=2/x^3 & & f'''(1)=2/1^3=2.
\end{array}
\]
And substituting into the polynomial equation we get
\[
p_{f,1}^3(x)=0+1(x-1)+\frac{-1}{2}(x-1)^2+\frac{2}{3!}(x-1)^3= \frac{2}{3}x^3-\frac{3}{2}x^2+3x-\frac{11}{6}.
\]
\end{frame}


%---------------------------------------------------------------------slide----
\begin{frame}
\frametitle{Taylor polynomials of the logarithmic function}
\begin{center}
\tikzsetnextfilename{derivatives_1_variable/taylor_polynomials_logarithm}
% Author: Alfredo Sánchez Alberca (asalber@ceu.es)
\begin{tikzpicture}
  \begin{axis}[
    2dfun, 
    xmin=0, xmax=4,
    ymin=-2, ymax=2,
    clip=false, 
    height=5cm,
  	]
    \addplot+[domain=0.2:2.8, smooth, samples=200] {ln(x)} node[anchor=west] {$f(x)=\log(x)$};
    \addplot+[domain=0.2:2.8, smooth, visible on=<2->] {x-1} node[anchor=west] {$p^1_{f,1}(x)=-1+x$};
    \addplot+[domain=0.2:2.8, smooth, visible on=<3->] {-1+x-(x-1)^2/2} node[anchor=west] {$p_{f,1}^2(x)=-1+x-\frac{1}{2}(x-1)^2$};
    \addplot+[domain=0.2:2.8, smooth, visible on=<4->] {-1+x-(x-1)^2/2+2/6*(x-1)^3} node[anchor=south west] {$p_{f,1}^3(x)=-1+x-\frac{1}{2}(x-1)^2+\frac{1}{3}(x-1)^3$};
    \coordinate (A) at (1,0);
    \fill (A) circle (1.2pt);
  \end{axis};
\end{tikzpicture}

\end{center}
\end{frame}


%---------------------------------------------------------------------slide----
\begin{frame}
\frametitle{Maclaurin polynomial of order $n$}
The Taylor polynomial equation simplifies when the polynomial is calculated at $0$.
This special case of Taylor polynomial at $0$ is known as the \emph{Maclaurin polynomial}.
\begin{definition}[Maclaurin polynomial]
Given a function $f(x)$ differentiable $n$ times at $0$, the \emph{Maclaurin polynomial} of order $n$ of $f$ is the polynomial with equation
\begin{align*}
p_{f,0}^n(x)&=f(0)+f'(0)x+\frac{f''(0)}{2}x^2+\cdots +\frac{f^{(n)}(0)}{n!}x^n = \\ &=\sum_{i=0}^{n}\frac{f^{(i)}(0)}{i!}x^i.
\end{align*}
\end{definition}
\end{frame}


%---------------------------------------------------------------------slide----
\begin{frame}
\frametitle{Maclaurin polynomial calculation}
\framesubtitle{Example}
Let's approximate the function $f(x)=\sin x$ near the value $0$ by a polynomial of order $3$.

The Maclaurin polynomial equation of order $3$ of $f$ is
\[
p_{f,0}^3(x)=f(0)+f'(0)x+\frac{f''(0)}{2}x^2+\frac{f'''(0)}{3!}x^3.
\]
The derivatives of $f$ at $0$ up to order $3$ are
\[
\begin{array}{lll}
f(x)=\sin x & \quad & f(0)=\sin 0 =0,\\
f'(x)=\cos x & & f'(0)=\cos 0=1,\\
f''(x)=-\sin x & & f''(0)=-\sin 0=0,\\
f'''(x)=-\cos x & & f'''(0)=-\cos 0=-1.
\end{array}
\]
And substituting into the polynomial equation we get
\[
p_{f,0}^3(x)=0+1\cdot x+\frac{0}{2}x^2+\frac{-1}{3!}x^3= x-\frac{x^3}{6}.
\]
\end{frame}


%---------------------------------------------------------------------slide----
\begin{frame}
\frametitle{Maclaurin polynomial of the sine function}
\begin{center}
\tikzsetnextfilename{derivatives_1_variable/maclaurin_polynomials_sine}
% Author: Alfredo Sánchez Alberca (asalber@ceu.es)
\begin{tikzpicture}[trim axis left, trim axis right]
  \begin{axis}[
    2dfun, 
    xmin=-3, xmax=3,
    ymin=-2, ymax=2,
    clip=false, 
    height=5cm,
  	]
    \addplot+[domain=-2.9:2.9, smooth, samples=200] {sin(deg(x))} node[anchor=west] {$f(x)=\sin(x)$};
    \addplot+[domain=-2:2, smooth, samples=200, visible on=<2->] {x} node[anchor=west] {$p^1_{f,0}(x)=x$};
    \addplot+[domain=-2.9:2.9, smooth, samples=200, visible on=<3->] {x-x^3/6} node[anchor=west] {$p_{f,0}^3(x)=x-\frac{1}{6}x^3$};
    \addplot+[domain=-2.9:2.9, smooth, samples=200, visible on=<4->] {x-x^3/6+x^5/120} node[anchor=west] {$p_{f,0}^5(x)=x-\frac{1}{6}x^3+\frac{1}{120}x^5$};
    \coordinate (A) at (0,0);
    \fill (A) circle (1.2pt);
  \end{axis};
\end{tikzpicture}

\end{center}
\end{frame}


%---------------------------------------------------------------------slide----
\begin{frame}
\frametitle{Maclaurin polynomials of elementary functions}
\[
\renewcommand{\arraystretch}{2.5}
\begin{array}{cc}
\toprule
f(x) & p_{f,0}^n(x) \\
\midrule
\sin x & \displaystyle x - \frac{x^3}{3!} + \frac{x^5}{5!} - \cdots + (-1)^k\frac{x^{2k-1}}{(2k-1)!} \mbox{ if $n=2k$ or $n=2k-1$}\\
\cos x &  \displaystyle 1 - \frac{x^2}{2!} + \frac{x^4}{4!} - \cdots + (-1)^k\frac{x^{2k}}{(2k)!} \mbox{ if $n=2k$ or $n=2k+1$}\\
\arctan x &  \displaystyle x - \frac{x^3}{3} + \frac{x^5}{5} - \cdots + (-1)^k\frac{x^{2k-1}}{(2k-1)} \mbox{ if $n=2k$ or $n=2k-1$}\\
e^x & \displaystyle 1 + x + \frac{x^2}{2!} + \frac{x^3}{3!} + \cdots + \frac{x^n}{n!}\\
\log(1+x) & \displaystyle x - \frac{x^2}{2} + \frac{x^3}{3} - \cdots + (-1)^{n-1}\frac{x^n}{n}\\
\bottomrule
\end{array}
\]
\end{frame}


%---------------------------------------------------------------------slide----
\begin{frame}
\frametitle{Taylor remainder and Taylor formula}
Taylor polynomials allow to approximate a function in a neighborhood of a value $x_0$, but there is always an error in the approxition.
\begin{definition}[Taylor remainder]
Given a function  $f(x)$ and its Taylor polynomial of order $n$ at $x_0$, $p_{f,x_0}^n(x)$, the \emph{Taylor remainder} of order $n$ of $f$ at $x_0$ is de difference between the function and the polynomial,
\[
r_{f,x_0}^n(x)=f(x)-p_{f,x_0}^n(x).
\]
\end{definition}

The Taylor remainder measures the error int the approximation of $f(x)$ by the Taylor polynomial and allow us to express the function as the Taylor polynomial plus the Taylor remainder
\[
f(x)=p_{f,x_0}^n(x) + r_{f,x_0}^n(x).
\]
This expression is known as \emph{Taylor formula} of order $n$ or $f$ at $x_0$. 

It can be proved that
\[
\lim_{h\rightarrow 0}\frac{r_{f,x_0}^n(x_0+h)}{h^n}=0,
\]
which means that the remainder $r_{f,x_0}^n(x_0+h)$ is much less than $h^n$.
\end{frame}
