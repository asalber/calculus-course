% Author: Alfredo Sánchez Alberca (asalber@ceu.es)

\section{Ordinary Differential Equations}

\mode<presentation>{
%---------------------------------------------------------------------slide----
\begin{frame}
\frametitle{Ordinary Differential Equations}
\tableofcontents[sectionstyle=show/hide,hideothersubsections]
\end{frame}
}


\subsection{Ordinary Differential Equations}
%---------------------------------------------------------------------slide----
\begin{frame}
\frametitle{Ordinary Differential Equation}
Often in Physics, Chemistry, Biology, Geometry, etc there arise equations that relate a function with its derivative, or successive derivatives.

\begin{definition}[Ordinary differential equation]
An \emph{ordinary differential equation} (O.D.E.) is a equation that relates an independent variable $x$, a function $y(x)$ that depends on $x$, and the successive derivatives of $y$, $y',y'',\ldots,y^{(n)}$; it can be written as
\[
F(x, y, y', y'',\ldots, y^{(n)})=0.
\] 
The \emph{order} of a differential equation is the greatest order of the derivatives in the equation. 
\end{definition}
Thus, for instance, the equation $y'''+sen(x)y'=2x$ is a differential equation of order 3.
\end{frame}


%---------------------------------------------------------------------slide----
\begin{frame}
\frametitle{Deducing a differential equation}
To deduce a differential equation that explains a natural phenomenon is essential to understand what a derivative is and how to interpret it. 

\structure{\textbf{Example}} Newton's law of cooling states
\begin{center}
\begin{minipage}{0.8\textwidth}
\textit{``The rate of change of the temperature of a body in a surrounding medium is proportional to the difference between the temperature of the body $T$ and the temperature of the medium $T_a$.''}
\end{minipage}
\end{center}
The rate of change of the temperature is the derivative of temperature with respect to time $dT/dt$.
Thus, Newton's law of cooling can be explained by the differential equation
\[
\frac{dT}{dt}=k(T-T_a),
\]
where $k$ is a proportionality constant.
\end{frame}


%---------------------------------------------------------------------slide----
\begin{frame}
\frametitle{Solution of an ordinary differential equation}
\begin{definition}[Solution of an ordinary differential equation]
Given an ordinary differential equation $F(x,y,y',y'',\ldots,y^{(n})=0$, the function $y=f(x)$ is a \emph{solution of the ordinary differential equation} if it satisfies the equation, that is, if
\[
F(x,f(x), f'(x), f''(x),\ldots, f^{(n}(x))=0.
\]
The graph of a solution of the ordinary differential equation is known as \emph{integral curve}.
\end{definition}

Solving an ordinary differential equations consists on finding all its solutions in a given domain. 
For integral calculus is required. 

The same manner than the indefinite integral is a family of antiderivatives, that differ in a constant term, after integrating an ordinary differential equation we get a family of solutions that differ in a constant.
We can get particular solutions by giving values to this constant.
\end{frame}


%---------------------------------------------------------------------slide----
\begin{frame}
\frametitle{General solution of an ordinary differential equation}
\begin{definition}[General solution of an ordinary differential equation]
Given an ordinary differential equation $F(x,y,y',y'',\ldots,y^{(n})=0$ of order $n$, the \emph{general solution} of the differential equation is a family of functions 
\[y =f (x,C_1,\ldots,C_n),\] 
depending on $n$ constants, such that for any value of $C_1,\ldots,C_n$ we get a solution of the differential equation. 
\end{definition}

For every value of the constant we get \emph{particular solution} of the differential equation.
Thus, when a differential equation can be solved, it has infinite solutions.

Geometrically, the general solution of a differential equation corresponds to a family of integral curves of the differential equation.

Often, it is common to impose conditions to the solutions of a differential equation to reduce the number of solutions. 
In many cases, these conditions allow to determine the values of the constants in the general solution to get a particular solution.
\end{frame}


%---------------------------------------------------------------------slide----
\begin{frame}
\frametitle{First order differential equations}
In this chapter we are going to study first order differential equations
\[
F(x,y,y')=0.
\]

The general solution of a first order differential equation is
\[
y = f (x,C),
\]
so to get a particular solution from the general one, it is enough to set the value of the constant $C$, and for that we only need to impose one initial condition.

\begin{definition}[Initial value problem]
The group consisting of a first order differential equation and an initial condition is known as \emph{initial value problem}:
\[
\begin{cases}
F(x,y,y')=0, & \mbox{First order differential equation;} \\
y(x_0)=y_0, & \mbox{Initial condition.}
\end{cases}
\]
\end{definition}

Solving an initial value problem consists in finding a solution of the first order differential equation that satisfies the initial condition.
\end{frame}


%---------------------------------------------------------------------slide----
\begin{frame}
\frametitle{Solving an initial value problem}
\framesubtitle{Example}
Recall the first order differential equation of the Newton's law of cooling,
\[
\frac{dT}{dt}=k(T-T_a),
\]
where $T$ is the temperature of the body and $T_a$ is the temperature of the surrounding medium.

It is easy to check that the general solution of this equation is 
\[T(t) = Ce^{kt}+T_a.\]

If we impose the initial condition that the temperature of the body at the initial instant is $5$ ºC, that is, $T(0)=5$, we have
\[T(0) = Ce^{k\cdot0}+T_a = C+T_a = 5,\]
from where we get $C=5-T_a$, and this give us the particular solution
\[
T(t) = (5-T_a)e^{kt}+T_a.
\]
\end{frame}


%---------------------------------------------------------------------slide----
\begin{frame}
\frametitle{Integral curve of an initial value problem}
\framesubtitle{Example}
If we assume in the previous example that the temperature of the surrounding medium is $T_a=0$ ºC and the cooling constant of the body is $k=1$, the general solution of the differential equation is
\[T(t)=Ce^t,\]
that is a family of integral curves.
Among all of them, only the one that passes through the point $(0,5)$ corresponds to the particular solution of the previous initial value problem. 
\begin{center}
\tikzsetnextfilename{ordinary_differential_equations/integral_curves}
% Author: Alfredo Sánchez Alberca (asalber@ceu.es)
\begin{tikzpicture}[trim axis left, trim axis right]
  \begin{axis}[
    2dfun, 
    xmin=-1, xmax=1,
    ymin=0, ymax=10,
    xlabel={$t$},         
    ylabel={$T$},      
    clip=false, 
    height=3cm,
  	]
  	\foreach \c in {1,2,3,4,6}{
  	\edef\temp{\noexpand \addplot[domain=-1:0.5, samples=100, smooth, thick, color1] {exp(x)*\c} node[anchor=west] {$T(t)=\c e^t$};}
  	\temp
    }
    \addplot[domain=-1:0.5, samples=100, smooth, thick, color2] {5*exp(x)} node[anchor=west] {$T(t)=5e^t$};
    \fill (0,5) circle (1.2pt);
  \end{axis};
\end{tikzpicture}

\end{center}
\end{frame}


%---------------------------------------------------------------------slide----
\begin{frame}
\frametitle{Existence and uniqueness of solutions}
\begin{theorem}[Existence and uniqueness of solutions of a first order ODE]
Given an initial value problem 
\[
\begin{cases}
y'=F(x,y);\\
y(x_0)=y_0;
\end{cases}
\]
if $F(x,y(x))$ is a function continuous on an open interval around the point $(x_0,y_0)$, then a solution of the initial value problem exists. 
If, in addition, $\frac{\partial F}{\partial y}$ is continuous in an open interval around $(x_0,y_0)$, the solution is unique. 
\end{theorem}

Although this theorem guarantees the existence and uniqueness of a solution of a first order differential equation, it does not provide a method to compute it. 
In fact, there is not a general method to solve first order differential equations, but we will see how to solve some types:
\begin{itemize}
\item Separable differential equations
\item Homogeneous differential equations
\item Linear differential equations
\end{itemize}
\end{frame}



\subsection{Separable differential equations}
%---------------------------------------------------------------------slide----
\begin{frame}
\frametitle{Separable differential equations}
\begin{definition}[Separable differential equation]
A \emph{separable differential equation} is a first order differential equation that can be written as
\[y'g(y)=f(x),\]
or what is the same,
\[g(y)dy=f(x)dx,\]
so the different variables are on different sides of the equality (the variables are separated).
\end{definition}

The general solution for a separable differential equation comes after integrating both sides of the equation
\[\int g(y)\,dy = \int f(x)\,dx+C.\]
\end{frame}


%---------------------------------------------------------------------slide----
\begin{frame}
\frametitle{Solving a separable differential equation}
\framesubtitle{Example}
The differential equation of the Newton's law of cooling 
\[\frac{dT}{dt}=k(T-T_a),\]
is a separable differential equation since it can be written as
\[\frac{1}{T-T_a}dT=k\,dt.\]

Integrating both sides of the equation we have
\[
\int \frac{1}{T-T_a}\,dT=\int k\,dt\Leftrightarrow \log(T-T_a)=kt+C,
\]
and solving for $T$ we get the general solution of the equation 
\[
T(t)=e^{kt+C}+T_a=e^Ce^{kt}+T_a=Ce^{kt}+T_a,
\]
rewriting $C=e^C$ as an arbitrary constant.
\end{frame}



\subsection{Homogeneous differential equations}
%---------------------------------------------------------------------slide----
\begin{frame}
\frametitle{Homogeneous functions}
\begin{definition}[Homogeneous function]
A function $f(x,y)$ is \emph{homogeneous} of degree $n$, if it satisfies
\[f(kx,ky)= k^nf(x,y),\]
for any value $k\in \mathbb{R}$.
\end{definition}

In particular, a homogeneous function of degree $0$ always satisfies
\[f(kx,ky)=f(x,y).\]

Setting $k=1/x$ we have
\[
f(x,y)=f\left(\frac{1}{x}x,\frac{1}{x}y\right)=f\left(1,\frac{y}{x}\right)=g\left(\frac{y}{x}\right).
\]
This way, a homogeneous function of degree $0$ always can be written as a function of $u=y/x$:
\[f(x,y)=g\left(\frac{y}{x}\right)=g(u).\]
\end{frame}


%---------------------------------------------------------------------slide----
\begin{frame}
\frametitle{Homogeneous differential equations}
\begin{definition}[Homogeneous differential equation]
A \emph{homogeneous differential equation} is a first order differential equation that can be written as 
\[y'=f(x,y),\]
where $f(x,y)$ is a homogeneous function of degree $0$.
\end{definition}

We can solve a homogeneous differential equation by making the substitution
\[
u=\frac{y}{x}\Leftrightarrow y=ux,
\]
so the equation becomes
\[
u'x+u=f(u),
\]
that is a separable differential equation.

Once solved the separable differential equation, the substitution must be undone.
\end{frame}


%---------------------------------------------------------------------slide----
\begin{frame}
\frametitle{Solving a homogeneous differential equation}
\framesubtitle{Example}
Let us consider the following differential equation
\[
4x-3y+y'(2y-3x)=0.
\]

Rewriting the equation in this way
\[
y'=\frac{3y-4x}{2y-3x}
\]
we can easily check that it is a homogeneous differential equation.

To solve this equation we have to do the substitution $y=ux$, and we get
\[
u'x+u=\frac{3ux-4x}{2ux-3x}=\frac{3u-4}{2u-3}
\]
that is a separable differential equation.

Separating the variables we have
\[
u'x=\frac{3u-4}{2u-3}-u=\frac{-2u^2+6u-4}{2u-3}\Leftrightarrow \frac{2u-3}{-2u^2+6u-4}\,du=\frac{1}{x}\,dx.
\]
\end{frame}


%---------------------------------------------------------------------slide----
\begin{frame}
\frametitle{Solving a homogeneous differential equation}
\framesubtitle{Example}
Now, integrating both sides of the equation we have
\[
\renewcommand{\arraystretch}{2}
\begin{array}{c}
\displaystyle \int \frac{2u-3}{-2u^2+6u-4}\,du=\int \frac{1}{x}\,dx
\Leftrightarrow -\frac{1}{2}\log|u^2-3u+2|=\log|x|+C \Leftrightarrow\\
\Leftrightarrow \log|u^2-3u+2|=-2\log|x|-2C,
\end{array}
\]
then, applying the exponential function to both sides and simplifying we get the general solution 
\[
u^2-3u+2=e^{-2\log|x|-2C}=\frac{e^{-2C}}{e^{\log|x|^2}}=\frac{C}{x^2},
\]
rewriting the constant $K=e^{-2C}$.

Finally, undoing the initial substitution $u=y/x$, we arrive at the general solution of the homogeneous differential equation
\[
\left(\frac{y}{x}\right)^2-3\frac{y}{x}+2=\frac{K}{x^2}\Leftrightarrow y^2-3xy+2x^2=K.
\]
\end{frame}



\subsection{Linear differential equations}
%---------------------------------------------------------------------slide----
\begin{frame}
\frametitle{Linear differential equations}
\begin{definition}[Linear differential equation]
A \emph{linear differential equation} is a first order differential equation that can be written as
\[y'+g(x)y = h(x).\]
\end{definition}

To solve a linear differential equation we try to write the left side of the equation as the derivative of a product. 
For that we multiply both sides by a function $f(x)$, such that
\[f'(x)=g(x)f(x).\]
Thus, we get
\[
\begin{array}{c}
y'f(x)+g(x)f(x)y=h(x)f(x)\\
\Updownarrow\\
y'f(x)+f'(x)y=h(x)f(x)\\
\Updownarrow\\
\dfrac{d}{dx}(yf(x))=h(x)f(x)
\end{array}
\]
\end{frame}


%---------------------------------------------------------------------slide----
\begin{frame}
\frametitle{Solving a linear differential equation}
Integrating both sides of the previous equation we get the solution
\[
yf(x)=\int h(x)f(x)\,dx+C.
\]
On the other hand, the unique function that satisfies $f'(x)=g(x)f(x)$ is
\[
f(x)=e^{\int g(x)\,dx},
\]
so, substituting this function in the previous solution we arrive at the solution of the linear differential equation
\[
ye^{\int g(x)\,dx}=\int h(x) e^{\int g(x)\,dx}\,dx+C,
\]
or what is the same
\[
y=e^{-\int g(x)\,dx}\left(\int h(x)e^{\int g(x)\,dx}\,dx+C\right).
\]
\end{frame}


%---------------------------------------------------------------------slide----
\begin{frame}
\frametitle{Solving a linear differential equation}
\framesubtitle{Example}
If in the differential equation of the Newton's law of cooling the temperature of the surrounding medium is a function of time $T_a(t)$, then the differential equation 
\[
\frac{dT}{dt}=k(T-T_a(t)),
\]
is a linear differential equation since it can be written as
\[
T'-kT=-kT_a(t),
\]
where the independent term is $-kT_a(t)$ and the coefficient of $T$ is $-k$.

Substituting in the formula of the general solution of a linear differential equation we have
\[
y=e^{-\int -k\,dt}\left(\int -kT_a(t)e^{\int -k\,dt}\,dt+C\right)=
e^{kt}\left(-\int kT_a(t)e^{-kt}\,dt+C\right).
\]
\end{frame}


%---------------------------------------------------------------------slide----
\begin{frame}
\frametitle{Solving a linear differential equation}
\framesubtitle{Example}
In the particular case that $T_a(t)=t$, and the proportionality constant $k=1$, the general solution of the linear differential equation is
\[
y=e^{t}\left(-\int te^{-kt}\,dt+C\right)=e^t(e^{-t}(t+1)+C)=Ce^t+t+1.
\]
If, in addition, we know that the temperature of the body at time $t=0$ is $5$ ºC, that is, we have the initial condition $T(0)=5$, then we can compute the value of the constant $C$,
\[
y(0)=Ce^0+0+1=5 \Leftrightarrow C+1=5 \Leftrightarrow C=4,
\]
and we get the particular solution
\[
y(t)=4e^t+t+1.
\]
\end{frame}
