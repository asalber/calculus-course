% arara: xelatex: {shell: true}
\documentclass{beamer}
% Remove navigation symbols
\setbeamertemplate{navigation symbols}{}%
% COLOR SETTINGS
% Author: Alfredo Sánchez Alberca (asalber@ceu.es)
\usepackage{xcolor}
\definecolor{color1}{RGB}{5,161,230} % Blue
\definecolor{color2}{RGB}{238,50,36} % Red
\definecolor{color3}{RGB}{0,205,0} % Green
\definecolor{color4}{RGB}{243,102,25} % Orange
\definecolor{ocre}{RGB}{243,102,25} % Define the orange color used for highlighting throughout the book
\definecolor{blueceu}{RGB}{5,161,230} % Blue color of CEU logo
\definecolor{greenceu}{RGB}{185,209,16} % Green color of CEU logo
\definecolor{redceu}{RGB}{238,50,36} % Red color of CEU logo
\definecolor{grayceu}{RGB}{111,107,83} % Gray color of CEU logo
\definecolor{coral}{rgb}{1,0.5,0.31} % Orange color for graphics
\definecolor{royalblue1}{rgb}{0.28,0.46,1} % Blue color for graphics
\definecolor{mygreen}{rgb}{0,0.8,0} % Green color for graphics
\definecolor{chaptergrey}{RGB}{5,161,230} % Blue color of CEU logo
\definecolor{DarkBrown}{HTML}{604c38} % Brown of Metropoly theme
\definecolor{DarkTeal}{HTML}{23373b} % Teal of Metropoly theme

% FONT SETTINGS
\usefonttheme{professionalfonts} % using non standard fonts for beamer
\usefonttheme{serif} % default family is serif
\usepackage{fontspec}
\usepackage{unicode-math}
\setmainfont[Ligatures=TeX]{TeX Gyre Pagella}
\setmathfont[math-style=ISO,bold-style=ISO,vargreek-shape=TeX]{TeX Gyre Pagella Math}


\usepackage{pgfplots}
\usepgfplotslibrary{fillbetween}
% PGFPLOTS SETTINGS
% Author: Alfredo Sánchez Alberca (asalber@ceu.es)

\pgfplotsset{colormap={outcolormap}{color=(color1); color=(color1!50)},
             colormap={incolormap}{color=(color2); color=(color2!50)},
             colormap={twocolormap}{color=(color1); color=(color2)}}

% Line styles for plotting 2d functions
\pgfplotscreateplotcyclelist{2dfun cycle}{%
  {color1, line width=0.8pt},
  {color2, line width=0.8pt},
  {color3, line width=0.8pt},
  {color4, line width=0.8pt},
}
% Line and point styles for plotting 2d data
\pgfplotscreateplotcyclelist{2dplot cycle}{%
  {color1, line width=0.8pt, mark=*, mark size=1.5pt},
  {color2, line width=0.8pt, mark=square*, mark size=1.3pt},
  {color3, line width=0.8pt, mark=triangle*, mark size=1.5pt},
  {color4, line width=0.8pt, mark=diamond*, mark size=1.5pt},
}

% Plot styles
\pgfplotsset{
  compat=1.14,
% Basic 2D settings
  2dbaseplot/.style={
    axis line style={<->}, % arrows on the axis
    scale only axis,        % otherwise width won't be as intended: http://tex.stackexchange.com/questions/36297/pgfplots-how-can-i-scale-to-text-width
    x tick label style={font=\small},
    y tick label style={font=\small},
    every axis/.append style={font=\small}, 
  },
% Settings for generic 2D functions 
  gen2dfun/.style={
    2dbaseplot,
    axis x line=middle,
    axis y line=middle,
    xlabel={$x$},         
    ylabel={$y$},          
    every axis x label/.style={at={(ticklabel* cs:1)}, anchor=west,},
	every axis y label/.style={at={(ticklabel* cs:1)}, anchor=south,},
    cycle list name=2dfun cycle,
  },
% Settings for 2D functions 
  2dfun/.style={
    2dbaseplot,
    axis x line=middle,
    axis y line=middle,
    xlabel={$x$},         
    ylabel={$y$},          
    every axis x label/.style={at={(ticklabel* cs:1)}, anchor=west,},
	every axis y label/.style={at={(ticklabel* cs:1)}, anchor=south,},
    x tick label style={font=\tiny},
    y tick label style={font=\tiny},
    cycle list name=2dfun cycle,
  },
% Settings for 2D plots
  2dplot/.style={
    baseplot,
    xmajorgrids=true,
    ymajorgrids=true,
    major grid style={dotted},
    axis x line=bottom,
    axis y line=left,
    legend style={cells={anchor=west}, draw=none, font=\footnotesize},
    x tick label style={font=\tiny}, 
    y tick label style={font=\tiny},
    cycle list name=2dplot cycle,
  },
% Basic 3D settings
  3dbaseplot/.style={
%    axis line style={->}, % arrows on the axis
    scale only axis,        % otherwise width won't be as intended: http://tex.stackexchange.com/questions/36297/pgfplots-how-can-i-scale-to-text-width
    x tick label style={font=\small},
    y tick label style={font=\small},
    z tick label style={font=\small},
    every axis/.append style={font=\small}, 
  },
% Settings for generic 3D functions 
  gen3dfun/.style={
    3dbaseplot,
    xlabel={$x$},         
    ylabel={$y$},
    zlabel={$z$},
    every axis x label/.style={at={(ticklabel cs:0.5)}, anchor=center,},
	every axis y label/.style={at={(ticklabel cs:0.5)}, anchor=center,},
	every axis z label/.style={at={(ticklabel cs:0.5)}, anchor=center,},           
    cycle list name=2dfun cycle,
  },
% Settings for 3D functions 
  3dfun/.style={
    3dbaseplot,
    xlabel={$x$},         
    ylabel={$y$},
    zlabel={$z$},           
    every axis x label/.style={at={(ticklabel cs:0.5)}, anchor=center,},
	every axis y label/.style={at={(ticklabel cs:0.5)}, anchor=center,},
	every axis z label/.style={at={(ticklabel cs:0.5)}, anchor=center,},
	x tick label style={font=\tiny}, 
    y tick label style={font=\tiny},
    z tick label style={font=\tiny},
    cycle list name=2dfun cycle,
  },
}

\usepackage{tikz}
\usetikzlibrary{external, arrows, arrows.meta, calc, shapes, shapes.arrows, positioning, decorations.pathreplacing, intersections}
% TIKZ SETTINGS
\tikzset{
	% Overlays
    invisible/.style={opacity=0},
    visible on/.style={alt={#1{}{invisible}}},
    alt/.code args={<#1>#2#3}{%
      \alt<#1>{\pgfkeysalso{#2}}{\pgfkeysalso{#3}} % \pgfkeysalso doesn't change the path
    },
    % Vectors
    vector/.style={->, color1, thick},
    % Arrow tips
    >=stealth,
  }

\tikzset{
    animation export/.style={
        external/system call={
            xelatex \tikzexternalcheckshellescape -halt-on-error -interaction=batchmode -jobname "\image" "\texsource";
            convert -verbose -delay 250 -loop 0 -density 300 -background white -alpha remove "\image.pdf" "\image.gif"
        }
    }
}

\tikzexternalize[prefix=img/exported/]
\tikzset{animation export}


\begin{document}
\begin{frame}
\begin{center}
\tikzsetnextfilename{derivatives_1_variable/tangent_line_approximation}
% Author: Alfredo Sánchez Alberca (asalber@ceu.es)
\begin{tikzpicture}
  \begin{axis}[
    gen2dfun, 
    xmin=0, xmax=4.5,
    ymin=0, ymax=3.5,
    xtick={1,3},
    xticklabels={$a$, $a+\Delta x$},
    ytick={1,2.5,1.693147},
    yticklabels={$f(a)$,$f(a+\Delta x)$, $f(a)+f'(a)\Delta x$},
    clip=false,
    height=4cm,
    ]
    \addplot+[domain=0.2:3.5, smooth] {2^(x-2)+0.5} node[anchor=south] {$f(x)$};
    \addplot+[domain=0.2:3.5, smooth] {1+ln(2)/2*(x-1)} node[anchor=south west] {Tangent} node [anchor=west] {$y=f(a)+f'(a)(x-a)$};
    \coordinate (O);
    \coordinate (A) at (1,1);
    \coordinate (B) at (3,2.5);
    \coordinate (C) at (3,1.693147);
    \fill (A) circle (1.2pt);
    \draw[gray, dotted] (A) -- (A|-O);
    \draw[gray, dotted] (A) -- (A-|O);
    \draw[gray, dotted] (B) -- (B|-O);
    \draw[gray, dotted] (B) -- (B-|O);
    \draw[gray, dotted] (C) -- (C-|O);
    \draw (C) -- (C|-A) node[midway, anchor=west] {$f'(a)\Delta x$};
    \draw (A) -- (A-|C) node[midway, anchor=north] {$\Delta x$};
    \draw[color3] (B) -- (C) node[midway, anchor=east] {Error};
  \end{axis}
\end{tikzpicture}

% \tikzsetnextfilename{derivatives_1_variable/approximation_polynomial_0}
% % Author: Alfredo Sánchez Alberca (asalber@ceu.es)
\begin{tikzpicture}
  \begin{axis}[
    gen2dfun, 
    xmin=0, xmax=4,
    ymin=0, ymax=4,
    xtick={2.5},
    xticklabel={$x_0$},
    ytick={1.5},
    yticklabels={$f(x_0)$}, 
    clip=false, 
    height=5cm,
  	]
    \addplot+[domain=0.5:3.6, smooth, name path=F] {2.7183^(x-2.5)+0.5} node[anchor=south west] {$f(x)$};
    \coordinate (A) at (2.5,1.5);
    \fill (A) circle (1.2pt);
    \draw[gray, dotted] (A) -- (A|-O);
    \draw[gray, dotted] (A) -- (A-|O);
    \addplot+[domain=0.5:3.6, smooth, name path=P, visible on=<2->] {1.5} node[anchor=west] {$p^0=f(x_0)$};
    \coordinate (B) at (3.5,3.2183);
    \draw<3->[gray, very thin] (3.5,0.06) -- (3.5,-0.06);
    \node<3-> at (3.5,-0.25) {$x_1$};
    \draw<3->[gray, very thin] (-0.06,3.2183) -- (0.06,3.2183);
    \node<3-> at (-0.25,3.2183) {$x_1$};
    \draw<3->[gray, dotted] (B) -- (B|-O);
    \draw<3->[gray, dotted] (B) -- (B-|O);
    \draw[|-|, visible on=<4->] (B) -- (B|-A) node[anchor=west, pos=0.3] {Approximation error} node[anchor=west, midway] {$e^0(x_1)=f(x_1)-p^0(x_1)$};
  \end{axis};
\end{tikzpicture}

% \tikzsetnextfilename{derivatives_1_variable/approximation_polynomial_1}
% % Author: Alfredo Sánchez Alberca (asalber@ceu.es)
\begin{tikzpicture}
  \begin{axis}[
    gen2dfun, 
    xmin=0, xmax=4,
    ymin=0, ymax=4,
    xtick={2.5},
    xticklabel={$x_0$},
    ytick={1.5},
    yticklabels={$f(x_0)$}, 
    clip=false, 
    height=5cm,
  	]
    \addplot+[domain=0.5:3.6, smooth, name path=F] {2.7183^(x-2.5)+0.5} node[anchor=south west] {$f(x)$};
    \coordinate (O);
    \coordinate (A) at (2.5,1.5);
    \fill (A) circle (1.2pt);
    \draw[gray, dotted] (A) -- (A|-O);
    \draw[gray, dotted] (A) -- (A-|O);
    \addplot+[domain=0.5:3.6, smooth, name path=P0, visible on=<2->] {1.5} node[anchor=west] {$p^0=f(x_0)$};
    \addplot+[domain=1:3.6, smooth, name path=P1, visible on=<3->] {x-1} node[anchor=west] {$p^1=f(x_0)+f'(x_0)(x-x_0)$};
    \coordinate (B) at (3.5,3.2183);
    \coordinate (C) at (3.5,2.5);
    \draw<4->[gray, very thin] (3.5,0.06) -- (3.5,-0.06);
    \node<4-> at (3.5,-0.25) {$x_1$};
    \draw<4->[gray, very thin] (-0.06,3.2183) -- (0.06,3.2183);
    \node<4->[anchor=east] at (-0.15,3.2183) {$f(x_1)$};
    \draw<4->[gray, very thin] (-0.06,2.5) -- (0.06,2.5);
    \node<4->[anchor=east] at (-0.15,2.5) {$p^1(x_1)$};
    \draw<4->[gray, dotted] (B) -- (B-|O);
    \draw<4->[gray, dotted] (B) -- (B|-O);
    \draw<4->[gray, dotted] (C) -- (C-|O);
    \draw[|-|, visible on=<5->] (B) -- (C) node[anchor=west, pos=0.1] {Approximation error} node[anchor=west, midway] {$e^1(x_1)=f(x_1)-p^1(x_1)$};
  \end{axis};
\end{tikzpicture}

% \tikzsetnextfilename{derivatives_1_variable/approximation_polynomial_2}
% % Author: Alfredo Sánchez Alberca (asalber@ceu.es)
\begin{tikzpicture}[trim axis left, trim axis right]
  \begin{axis}[
    gen2dfun, 
    xmin=0, xmax=4,
    ymin=0, ymax=4,
    xtick={2.5},
    xticklabel={$a$},
    ytick={1.5},
    yticklabels={$f(a)$}, 
    clip=false, 
    height=5cm,
  	]
    \addplot+[domain=0.5:3.6, smooth, name path=F] {2.7183^(x-2.5)+0.5} node[anchor=south west] {$f(x)$};
    \coordinate (O);
    \coordinate (A) at (2.5,1.5);
    \fill (A) circle (1.2pt);
    \draw[gray, dotted] (A) -- (A|-O);
    \draw[gray, dotted] (A) -- (A-|O);
    \addplot+[domain=0.5:3.6, smooth, name path=P0, visible on=<2->] {1.5} node[anchor=west] {$p^0=f(a)$};
    \addplot+[domain=1:3.6, smooth, name path=P1, visible on=<3->] {x-1} node[anchor=west] {$p^1=f(a)+f'(a)(x-a)$};
    \addplot+[domain=0.5:3.6, smooth, name path=P1, visible on=<4->] {-1+x+(x-2.5)^2/2} node[anchor=west] {$p^2=f(a)+f'(a)(x-a)+\frac{f''(a)}{2}(x-a)^2$};
    \coordinate (B) at (3.5,3.2183);
    \coordinate (C) at (3.5,3);
    \draw<4->[gray, very thin] (3.5,0.05) -- (3.5,-0.05);
    \node<4-> at (3.5,-0.25) {$x$};
    \draw<4->[gray, very thin] (-0.06,3.2183) -- (0.06,3.2183);
    \node<4-> at (-0.25,3.2183) {$f(x)$};
    \draw<4->[gray, dotted] (B) -- (B|-O);
    \draw<4->[gray, dotted] (B) -- (B-|O);
    \draw[dashed, visible on=<5->] (B) -- (B|-A);
    \draw[decorate, decoration={brace, amplitude=4pt}, visible on=<5->] (B) -- (B|-A) node [pos=0.3, right, xshift=5pt] {Approximation error} node[midway, right, xshift=4pt] {$e^2(x)=f(x)-p^2(x)$};
  \end{axis};
\end{tikzpicture}

% \tikzsetnextfilename{derivatives_1_variable/taylor_polynomials_logarithm}
% % Author: Alfredo Sánchez Alberca (asalber@ceu.es)
\begin{tikzpicture}
  \begin{axis}[
    2dfun, 
    xmin=0, xmax=4,
    ymin=-2, ymax=2,
    clip=false, 
    height=5cm,
  	]
    \addplot+[domain=0.2:2.8, smooth, samples=200] {ln(x)} node[anchor=west] {$f(x)=\log(x)$};
    \addplot+[domain=0.2:2.8, smooth, visible on=<2->] {x-1} node[anchor=west] {$p^1_{f,1}(x)=-1+x$};
    \addplot+[domain=0.2:2.8, smooth, visible on=<3->] {-1+x-(x-1)^2/2} node[anchor=west] {$p_{f,1}^2(x)=-1+x-\frac{1}{2}(x-1)^2$};
    \addplot+[domain=0.2:2.8, smooth, visible on=<4->] {-1+x-(x-1)^2/2+2/6*(x-1)^3} node[anchor=south west] {$p_{f,1}^3(x)=-1+x-\frac{1}{2}(x-1)^2+\frac{1}{3}(x-1)^3$};
    \coordinate (A) at (1,0);
    \fill (A) circle (1.2pt);
  \end{axis};
\end{tikzpicture}

% \tikzsetnextfilename{derivatives_1_variable/maclaurin_polynomials_sine}
% % Author: Alfredo Sánchez Alberca (asalber@ceu.es)
\begin{tikzpicture}[trim axis left, trim axis right]
  \begin{axis}[
    2dfun, 
    xmin=-3, xmax=3,
    ymin=-2, ymax=2,
    clip=false, 
    height=5cm,
  	]
    \addplot+[domain=-2.9:2.9, smooth, samples=200] {sin(deg(x))} node[anchor=west] {$f(x)=\sin(x)$};
    \addplot+[domain=-2:2, smooth, samples=200, visible on=<2->] {x} node[anchor=west] {$p^1_{f,0}(x)=x$};
    \addplot+[domain=-2.9:2.9, smooth, samples=200, visible on=<3->] {x-x^3/6} node[anchor=west] {$p_{f,0}^3(x)=x-\frac{1}{6}x^3$};
    \addplot+[domain=-2.9:2.9, smooth, samples=200, visible on=<4->] {x-x^3/6+x^5/120} node[anchor=west] {$p_{f,0}^5(x)=x-\frac{1}{6}x^3+\frac{1}{120}x^5$};
    \coordinate (A) at (0,0);
    \fill (A) circle (1.2pt);
  \end{axis};
\end{tikzpicture}

% \tikzsetnextfilename{integrals/riemann_sums}
% % Author: Alfredo Sánchez Alberca (asalber@ceu.es)
\begin{tikzpicture}[/pgf/declare function={f=x^3-6*x^2+11*x-3;}, trim axis left, trim axis right]
\begin{axis}[
	gen2dfun,
	xmin=0, xmax=4,
	ymin=0, ymax=4,
	xtick={1,3},
	xticklabels={$a$,$b$},
	ytick={},
	yticklabels={},
    clip=false,
    width=4cm,
    height=3cm,
	]
	\addplot+[domain=0.7:3.3, samples=100, smooth, name path=F] {f} node[anchor=south] {$f(x)$};
	\foreach \x in {2,...,10}{
		\only<\x>{\addplot[ybar interval, fill=color1, opacity=0.3, domain=1:3, samples=\x] {f};}
	 	\pgfmathsetmacro{\coor}{1+2/(\x-1)}
     	\edef\temp{\noexpand\draw[decorate, decoration={brace, amplitude=3pt, mirror}, visible on=<\x>] (1,0) -- (\coor,0) node [midway, below, yshift=-2pt] {$\Delta x$};}
    	\temp
	}
	\draw[decorate, decoration={brace, amplitude=4pt}, visible on=<2-10>] (1,0) -- (1,3) node [midway, left, xshift=-2pt] {$f(x_i)$};
	\only<11>{\addplot[ybar interval, color1, fill=color1, opacity=0.3, domain=1:3, samples=50] {f};}
	\draw[decorate, decoration={brace, amplitude=4pt}, visible on=<11>] (1,0) -- (1,3) node [midway, left, xshift=-2pt] {$f(x)$};
	\node[below, visible on=<11>] at (2,0) {$dx$};
	\node[visible on=<12->] at (2,1.5) {$\displaystyle \int_a^b f(x)\,dx$}; 
	\path[name path=axis] (axis cs:1,0) -- (axis cs:3,0);
    \addplot [thick, color=color1, fill=color1, fill opacity=0.3, visible on=<12->] fill between[of=F and axis, soft clip={domain=1:3},];
\end{axis}
\end{tikzpicture}

\end{center}
\end{frame}
\end{document}
